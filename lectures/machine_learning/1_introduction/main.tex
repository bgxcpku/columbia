\documentclass{beamer}

% \usepackage{beamerthemesplit} // Activate for custom appearance

\title{Introduction}
\author{Frank Wood}
\date{\today}

% !TEX root = talk.tex

\newcommand{\comment}[1]{}
%\newcommand{\comment}[1]{{\marginpar{\tiny {#1} }}}
\def\todo#1{TODO(#1)}

\def\bigO{{\mathcal O}}
\def\balpha{\mbox{\boldmath $\alpha$}}
\def\bbeta{\mbox{\boldmath $\beta$}}
\def\beeta{\mbox{\boldmath $\eta$}}
\def\blambda{\mbox{\boldmath $\lambda$}}
\def\bmu{\mbox{\boldmath $\mu$}}
\def\bphi{\mbox{\boldmath $\phi$}}
\def\bpsi{\mbox{\boldmath $\psi$}}
\def\bsigma{\mbox{\boldmath $\sigma$}}
\def\btau{\mbox{\boldmath $\tau$}}
\def\btheta{\mbox{\boldmath $\theta$}}
\def\dbphi{\dot{\mbox{\boldmath $\phi$}}}
\def\dbtau{\dot{\mbox{\boldmath $\tau$}}}
\def\dbtheta{\dot{\mbox{\boldmath $\theta$}}}

%\newcommand{\nofootnotemark}{\let\@makefnmark\relax}
\newcommand{\bX}{\mathbf{X}}
\newcommand{\bY}{\mathbf{Y}}
\newcommand{\bW}{\mathbf{W}}
\newcommand{\bZ}{\mathbf{Z}}
\newcommand{\bH}{\mathbf{H}}
\newcommand{\bQ}{\mathbf{Q}}
\newcommand{\bA}{\mathbf{A}}
\newcommand{\bI}{\mathbf{I}}
\newcommand{\by}{\mathbf{y}}
\newcommand{\bz}{\mathbf{z}}
\newcommand{\bx}{\mathbf{x}}

\newcommand{\ith}{i^\mathrm{th}}
\def\A{{\bf A}}
\def\B{{\bf B}}
\def\C{{\bf C}}
\def\D{{\bf D}}
\def\F{{\bf F}}
\def\L{{\bf L}}
\def\M{{\bf M}}
\def\W{{\bf W}}
\def\I{{\bf I}}
\def\J{{\bf J}}
\def\R{{\bf R}}
\def\U{{\bf U}}
\def\V{{\bf V}}
\def\b{{\bf b}}
\def\c{{\bf c}}
\def\d{{\bf d}}
\def\r{{\bf r}}
\def\s{{\bf s}}
\def\t{{\bf t}}
\def\u{{\bf u}}
\def\v{{\bf v}}
\def\f{{\bf f}}
\def\x{{\bf x}}
\def\y{{\bf y}}
\def\w{{\bf w}}
\def\vo{{\bf o}}
\def\p{{\bf p}}
\def\O{{\bf 0}}
%\def\a{{\bf a}}


\def\vbpsi{\vec{\mbox{\boldmath $\psi$}}} 
\def\vpsi{\vec{\psi}} 
\def\vbphi{\vec{\mbox{\boldmath $\phi$}}} 
\def\vphi{\vec{\phi}} 
\def\vbtau{\vec{\mbox{\boldmath $\tau$}}} 
\def\vbtheta{\vec{\mbox{\boldmath $\theta$}}} 
\def\vD{\vec{D}}
\def\vf{\vec{\bf f}}
\def\vF{\vec{\bf F}}
\def\vI{\vec{\bf I}}
\def\vR{\vec{\bf R}}
\def\vv{\vec{v}}
\def\vV{\vec{\bf V}}

\def\pon{p_{\mathrm{on}}}
\def\poff{p_{\mathrm{off}}}

\def\tr{^{\text{T}}}

%%% Vector notation for sections 3 and 4
%%% Vector notation for sections 3 and 4
\def\mvec{\vec{m}}
\def\fvec{\vec{f}}
\def\appfvec{\vec{f}_k}
\def\avec{\vec{a}}
\def\bvec{\vec{b}}
\def\evec{\vec{e}}
\def\uvec{\vec{u}}
\def\xvec{\vec{x}}
\def\wvec{\vec{w}}
\def\gradvec{\vec{\nabla}}

\def\aM{\mbox{\bf a}_M}
\def\aS{\mbox{\bf a}_S}
\def\aO{\mbox{\bf a}_O}
\def\aL{\mbox{\bf a}_L}
\def\aP{\mbox{\bf a}_P}
\def\ai{\mbox{\bf a}_i}
\def\aj{\mbox{\bf a}_j}
\def\an{\mbox{\bf a}_n}
\def\a1{\mbox{\bf a}_1}
\def\a2{\mbox{\bf a}_2}
\def\a3{\mbox{\bf a}_3}
\def\a4{\mbox{\bf a}_4}

%\def\x{\mbox{\bf x\/}}
%\def\X{\mbox{\bf X}}
%\def\A{\mbox{\bf A}}
%\def\P{\mbox{\bf P}}
%\def\C{\mbox{\bf C}}
%\def\c{\mbox{\bf c}}
%\def\b{\mbox{\bf b}}
%\def\o{\mbox{\bf o}}
%\def\h{\mbox{\bf h}}
%\def\f{\mbox{\bf f}}
%\def\x{\mbox{\bf x}}
%\def\sx{\mbox{\scriptsize\bf x}}
%\def\z{\mbox{\bf z}}
%\def\l{\mbox{\bf l}}
%\def\m{\mbox{\bf m}}
%\def\bi{\mbox{\bf i}}
%\def\u{\mbox{\bf u}}
%\def\v{\mbox{\bf v}}
\def\a{\mbox{\bf a}}
%\def\p{\mbox{\bf p}}
%\def\r{\mbox{\bf r}}
%\def\d{\mbox{\bf d}}
%\def\Q{\mbox{\bf Q}}
%\def\s{\mbox{\bf s}}
%\def\st{\mbox{\scriptsize\bf t}}
%\def\ss{\mbox{\scriptsize\bf s}}
%\def\t{\mbox{\bf t}}
%\def\cR{{\cal R}}
%\def\calD{{\cal D}}
%\def\calS{{\cal S}}
%\def\g{\mbox{\bf g}}
%\def\e{\mbox{\bf e}}
%\def\flow{\{\mbox{\bf u}\}}
%\def\appearChange{iconic change}

\def\sigmae{\sigma}
\def\sigmam{\sigma}

\newcommand{\eg}{e.\thinspace{}g.,\@\xspace}
\newcommand{\egn}{e.\thinspace{}g.\@\xspace}
\newcommand{\cf}{cf.\@\xspace}
\newcommand{\ie}{i.\thinspace{}e.,\@\xspace}
\newcommand{\ien}{i.\thinspace{}e.\@\xspace}
\newcommand{\iid}{i.\thinspace{}i.\thinspace{}d.\@\xspace}


%\newcommand{\comment}[1]{}
\newcommand{\ponedec}{\mathcal{P}^\downarrow_1}
\newcommand{\pone}{\mathcal{P}_1}
\newcommand{\rank}[1]{\mathrm{RANK}\left[#1\right]}
\newcommand{\E}[1]{\mathrm{E}\left[#1\right]}
%\newcommand{\PY}{\mathcal{PY}}
%\newcommand{\DP}{\mathcal{DP}}
%\newcommand{\iid}{iid.}
\newcommand{\drawiid}{\stackrel{\text{iid}}{\sim}}
\newcommand{\vect}[1]{\mathbf{#1}}
\newcommand{\indicator}[1]{\text{I}\left[ #1 \right]}
\newcommand{\pdcoag}{PD(d_1,0)-\text{COAG}}
%\newcommand{\todo}{\textbf{*TODO*}}
\newcommand{\igram}{\text{$\infty$-gram}}
\newcommand{\Prob}{\text{P}}

\def\mm{sequence memoizer }
\def\MM{SM }

\def\pibf{{\boldsymbol{\pi}}}
\def\kapbf{\boldsymbol{\kappa}}
\def\taubf{\boldsymbol{\tau}}
\def\thebf{\boldsymbol{\theta}}
\def\rhobf{\boldsymbol{\rho}}
\def\phibf{\boldsymbol{\phi}}
\def\pbf{\mathbf{p}}
\def\qbf{\mathbf{q}}
\def\sbf{\mathbf{s}}
\def\tbf{\mathbf{t}}
\def\ybf{\mathbf{y}}
\def\ubf{\mathbf{u}}
\def\Ave{\mathbb{E}}

\def\wbf{\mathbf{w}}
\def\xbf{\mathbf{x}}
\def\rbf{\mathbf{r}}
\def\tbf{\mathbf{t}}
\def\kbf{\mathbf{k}}
\def\Xbf{\mathbf{X}}
\def\0bf{\mathbf{0}}
\def\Ibf{\mathbf{I}}
\def\phibf{\mathbf{\phi}}
\def\Phibf{\mathbf{\Phi}}
\def\disteq{{\stackrel{D}{=}}}
\def\EE{{\mathbb{E}}}
\def\GG{\mathcal{G}}
\def\G{G}
\def\U{U}

\def\phiv{\varphi}
\def\phivbf{\boldsymbol{\varphi}}

\def\Ocal{\mathcal{O}}
\DeclareMathOperator*{\Var}{Var}

\DeclareMathOperator*{\Bet}{Beta}
\DeclareMathOperator{\coag}{COAG}
\DeclareMathOperator{\frag}{FRAG}
\DeclareMathOperator*{\rnk}{RANK}
\DeclareMathOperator*{\gem}{GEM}
\DeclareMathOperator*{\pd}{PD}
\DeclareMathOperator*{\py}{PY}
\DeclareMathOperator*{\DP}{DP}
\DeclareMathOperator*{\PY}{PY}
\DeclareMathOperator*{\gd}{GDir}
\DeclareMathOperator*{\Dir}{Dir}
\DeclareMathOperator*{\CRP}{CRP}
\DeclareMathOperator*{\argmax}{argmax}



%%% Local Variables: 
%%% mode: latex
%%% TeX-master: "paper"
%%% End: 
% !TEX root = talk.tex
%
%\newcommand{\comment}[1]{}
%%\newcommand{\comment}[1]{{\marginpar{\tiny {#1} }}}
%
%\def\bigO{{\mathcal O}}
%\def\balpha{\mbox{\boldmath $\alpha$}}
%\def\bbeta{\mbox{\boldmath $\beta$}}
%\def\beeta{\mbox{\boldmath $\eta$}}
%\def\blambda{\mbox{\boldmath $\lambda$}}
%\def\bmu{\mbox{\boldmath $\mu$}}
%\def\bphi{\mbox{\boldmath $\phi$}}
%\def\bpsi{\mbox{\boldmath $\psi$}}
%\def\bsigma{\mbox{\boldmath $\sigma$}}
%\def\btau{\mbox{\boldmath $\tau$}}
%\def\btheta{\mbox{\boldmath $\theta$}}
%\def\dbphi{\dot{\mbox{\boldmath $\phi$}}}
%\def\dbtau{\dot{\mbox{\boldmath $\tau$}}}
%\def\dbtheta{\dot{\mbox{\boldmath $\theta$}}}
%
%%\newcommand{\nofootnotemark}{\let\@makefnmark\relax}
%\newcommand{\bX}{\mathbf{X}}
%\newcommand{\bY}{\mathbf{Y}}
%\newcommand{\bW}{\mathbf{W}}
%\newcommand{\bZ}{\mathbf{Z}}
%\newcommand{\bH}{\mathbf{H}}
%\newcommand{\bQ}{\mathbf{Q}}
%\newcommand{\bA}{\mathbf{A}}
%\newcommand{\bI}{\mathbf{I}}
%\newcommand{\by}{\mathbf{y}}
%\newcommand{\bz}{\mathbf{z}}
%\newcommand{\bx}{\mathbf{x}}
%
%\newcommand{\ith}{i^\mathrm{th}}
%\def\A{{\bf A}}
%\def\B{{\bf B}}
%\def\C{{\bf C}}
%\def\D{{\bf D}}
%\def\F{{\bf F}}
%\def\L{{\bf L}}
%\def\M{{\bf M}}
%\def\W{{\bf W}}
%\def\I{{\bf I}}
%\def\J{{\bf J}}
%\def\R{{\bf R}}
%\def\U{{\bf U}}
%\def\V{{\bf V}}
%\def\b{{\bf b}}
%\def\c{{\bf c}}
%\def\d{{\bf d}}
%\def\r{{\bf r}}
%\def\s{{\bf s}}
%\def\t{{\bf t}}
%\def\u{{\bf u}}
%\def\v{{\bf v}}
%\def\f{{\bf f}}
%\def\x{{\bf x}}
%\def\y{{\bf y}}
%\def\w{{\bf w}}
%\def\vo{{\bf o}}
%\def\p{{\bf p}}
%\def\O{{\bf 0}}
%%\def\a{{\bf a}}
%
%
%\def\vbpsi{\vec{\mbox{\boldmath $\psi$}}} 
%\def\vpsi{\vec{\psi}} 
%\def\vbphi{\vec{\mbox{\boldmath $\phi$}}} 
%\def\vphi{\vec{\phi}} 
%\def\vbtau{\vec{\mbox{\boldmath $\tau$}}} 
%\def\vbtheta{\vec{\mbox{\boldmath $\theta$}}} 
%\def\vD{\vec{D}}
%\def\vf{\vec{\bf f}}
%\def\vF{\vec{\bf F}}
%\def\vI{\vec{\bf I}}
%\def\vR{\vec{\bf R}}
%\def\vv{\vec{v}}
%\def\vV{\vec{\bf V}}
%
%\def\pon{p_{\mathrm{on}}}
%\def\poff{p_{\mathrm{off}}}
%
%\def\tr{^{\text{T}}}
%
%%%% Vector notation for sections 3 and 4
%%%% Vector notation for sections 3 and 4
%\def\mvec{\vec{m}}
%\def\fvec{\vec{f}}
%\def\appfvec{\vec{f}_k}
%\def\avec{\vec{a}}
%\def\bvec{\vec{b}}
%\def\evec{\vec{e}}
%\def\uvec{\vec{u}}
%\def\xvec{\vec{x}}
%\def\wvec{\vec{w}}
%\def\gradvec{\vec{\nabla}}
%
%\def\aM{\mbox{\bf a}_M}
%\def\aS{\mbox{\bf a}_S}
%\def\aO{\mbox{\bf a}_O}
%\def\aL{\mbox{\bf a}_L}
%\def\aP{\mbox{\bf a}_P}
%\def\ai{\mbox{\bf a}_i}
%\def\aj{\mbox{\bf a}_j}
%\def\an{\mbox{\bf a}_n}
%\def\a1{\mbox{\bf a}_1}
%\def\a2{\mbox{\bf a}_2}
%\def\a3{\mbox{\bf a}_3}
%\def\a4{\mbox{\bf a}_4}
%
%%\def\x{\mbox{\bf x\/}}
%%\def\X{\mbox{\bf X}}
%%\def\A{\mbox{\bf A}}
%%\def\P{\mbox{\bf P}}
%%\def\C{\mbox{\bf C}}
%%\def\c{\mbox{\bf c}}
%%\def\b{\mbox{\bf b}}
%%\def\o{\mbox{\bf o}}
%%\def\h{\mbox{\bf h}}
%%\def\f{\mbox{\bf f}}
%%\def\x{\mbox{\bf x}}
%%\def\sx{\mbox{\scriptsize\bf x}}
%%\def\z{\mbox{\bf z}}
%%\def\l{\mbox{\bf l}}
%%\def\m{\mbox{\bf m}}
%%\def\bi{\mbox{\bf i}}
%%\def\u{\mbox{\bf u}}
%%\def\v{\mbox{\bf v}}
%\def\a{\mbox{\bf a}}
%%\def\p{\mbox{\bf p}}
%%\def\r{\mbox{\bf r}}
%%\def\d{\mbox{\bf d}}
%%\def\Q{\mbox{\bf Q}}
%%\def\s{\mbox{\bf s}}
%%\def\st{\mbox{\scriptsize\bf t}}
%%\def\ss{\mbox{\scriptsize\bf s}}
%%\def\t{\mbox{\bf t}}
%%\def\cR{{\cal R}}
%%\def\calD{{\cal D}}
%%\def\calS{{\cal S}}
%%\def\g{\mbox{\bf g}}
%%\def\e{\mbox{\bf e}}
%%\def\flow{\{\mbox{\bf u}\}}
%%\def\appearChange{iconic change}
%
%\def\sigmae{\sigma}
%\def\sigmam{\sigma}
%
%\newcommand{\eg}{e.\thinspace{}g.,\@\xspace}
%\newcommand{\egn}{e.\thinspace{}g.\@\xspace}
%\newcommand{\cf}{cf.\@\xspace}
%\newcommand{\ie}{i.\thinspace{}e.,\@\xspace}
%\newcommand{\ien}{i.\thinspace{}e.\@\xspace}
%\newcommand{\iid}{i.\thinspace{}i.\thinspace{}d.\@\xspace}
%
%
%%\newcommand{\comment}[1]{}
%\newcommand{\ponedec}{\mathcal{P}^\downarrow_1}
%\newcommand{\pone}{\mathcal{P}_1}
%\newcommand{\rank}[1]{\mathrm{RANK}\left[#1\right]}
%%\newcommand{\E}[1]{\mathrm{E}\left[#1\right]}
%%\newcommand{\PY}{\mathcal{PY}}
%%\newcommand{\DP}{\mathcal{DP}}
%%\newcommand{\iid}{iid.}
%\newcommand{\drawiid}{\stackrel{\text{iid}}{\sim}}
%\newcommand{\vect}[1]{\mathbf{#1}}
%\newcommand{\indicator}[1]{\text{I}\left[ #1 \right]}
%\newcommand{\pdcoag}{PD(d_1,0)-\text{COAG}}
%%\newcommand{\todo}{\textbf{*TODO*}}
%\newcommand{\igram}{\text{$\infty$-gram}}
%\newcommand{\Prob}{\text{P}}
%
%\def\mm{sequence memoizer }
%\def\MM{SM }
%
%\def\pibf{{\boldsymbol{\pi}}}
%\def\kapbf{\boldsymbol{\kappa}}
%\def\taubf{\boldsymbol{\tau}}
%\def\thebf{\boldsymbol{\theta}}
%\def\rhobf{\boldsymbol{\rho}}
%\def\phibf{\boldsymbol{\phi}}
%\def\pbf{\mathbf{p}}
%\def\qbf{\mathbf{q}}
%\def\sbf{\mathbf{s}}
%\def\tbf{\mathbf{t}}
%\def\ybf{\mathbf{y}}
%\def\ubf{\mathbf{u}}
%\def\Ave{\mathbb{E}}
%
%\def\wbf{\mathbf{w}}
%\def\xbf{\mathbf{x}}
%\def\rbf{\mathbf{r}}
%\def\tbf{\mathbf{t}}
%\def\kbf{\mathbf{k}}
%\def\Xbf{\mathbf{X}}
%\def\0bf{\mathbf{0}}
%\def\Ibf{\mathbf{I}}
%\def\phibf{\mathbf{\phi}}
%\def\Phibf{\mathbf{\Phi}}
%\def\disteq{{\stackrel{D}{=}}}
%\def\GG{\mathcal{G}}
%\def\G{G}
%\def\U{U}
%
%\def\phiv{\varphi}
%\def\phivbf{\boldsymbol{\varphi}}
%
%\def\Ocal{\mathcal{O}}
%\DeclareMathOperator*{\Var}{Var}
%
%\DeclareMathOperator*{\Bet}{Beta}
%\DeclareMathOperator{\coag}{COAG}
%\DeclareMathOperator{\frag}{FRAG}
%\DeclareMathOperator*{\rnk}{RANK}
%\DeclareMathOperator*{\gem}{GEM}
%\DeclareMathOperator*{\pd}{PD}
%\DeclareMathOperator*{\py}{PY}
%\DeclareMathOperator*{\DP}{DP}
%\DeclareMathOperator*{\PY}{PY}
%\DeclareMathOperator*{\gd}{GDir}
%\DeclareMathOperator*{\Dir}{Dir}
%\DeclareMathOperator*{\CRP}{CRP}
%\DeclareMathOperator*{\argmax}{argmax}
%
\def\GG{\mathcal{G}}
\def\data{\mathbf{x}}
%\def\EE{\mathbb{E}}
\def\disc{d}
%\newcommand{\delete}[1]{} %\textcolor{red}{#1}
%\newcommand{\rewrite}[1]{#1}%{\textcolor{blue}{#1}} %
%\newcommand{\lambdabf}{\boldsymbol{\lambda}}
%\newcommand{\vbf}{\mathbf{v}}
%\newcommand{\Psmooth}{\Prob_\text{smooth}}
%%\newcommand{\parent}{\pi}
%\newcommand{\suffix}{\sigma}
%\newcommand{\UHPYP}{SM}
%\newcommand{\PLUMP}{PLUMP}
%\newcommand{\Oh}{\mathcal{O}}
%\newcommand{\tree}{\mathcal{T}}
\newcommand{\cct}{\hat{\mathcal{T}}}
\newcommand{\cctx}{\cct(\data)}
\newcommand{\Gu}{G_{\ubf}}
%\newcommand{\GuSet}{\{G_{\ubf}\}_{\ubf \in \Sigma^*}}
%\newcommand{\E}{\mathrm{E}}
%\newcommand{\UpdatePath}{\text{\textsc{UpdatePath}}}
%\newcommand{\Path}{\ensuremath{(\ubf_0,\ldots,\ubf_P)}}
%\newcommand{\PathProbability}{\text{\textsc{PathProbability}}}
%\newcommand{\TT}{\mathcal{T}}
%\newcommand{\ral}[1]{\stackrel{\mathtt{#1}}{\rightarrow}}
\def\parent{{\sigma(\mathbf{u})}}
%
%%\def\newblock{\hskip .11em plus .33em minus .07em}
%
%
%% \newcommand{\cusk}{c_{\ubf s k}}
%% \newcommand{\cus}{c_{\ubf s \cdot}}
%% \newcommand{\cu}{c_{\ubf \cdot \cdot}}
%% \newcommand{\tus}{t_{\ubf s}}
%% \newcommand{\tu}{t_{\ubf \cdot}}
%\newcommand{\cusk}{c_{\ubf s k}}
%\newcommand{\cus}{c_{\ubf s}}
%\newcommand{\cu}{c_{\ubf \cdot}}
%\newcommand{\tus}{t_{\ubf s}}
%\newcommand{\tu}{t_{\ubf \cdot}}
%\newcommand{\cset}{\{\cusk\}_{s\in \Sigma,k \in \{1,\ldots,t_{\ubf s}\}}}
%\newcommand{\tset}{\{\tus\}_{s\in \Sigma}}
%\newcommand{\bydef}{\equiv}
%\newcommand{\state}{\mathcal{S}_{\xbf}}
%\newcommand{\statei}{\mathcal{S}_{\xbf_{1:i}}}
%%\newcommand{\emptystring}{\varepsilon}
%\newcommand{\gcount}{\hat{c}}
%\newcommand{\escape}{\mathtt{esc}}
%\def\prob{G}
%
%
%\newcommand{\todo}[1]{\begin{center}\textbf{TODO: } #1 \end{center}}
%\newcommand{\figref}[1]{\figurename~\ref{#1}}
%\newcommand{\predictive}{\Prob(x_i|\xbf_{1:i-1})}
%\newcommand{\ywcomment}[1]{\textbf{#1}}
%\newcommand{\jgcomment}[1]{ { \textcolor{red}{#1} } }
%
%\newcommand{\secref}[1]{Section \ref{#1}}
%
\def\context{\mathbf{u}}

%%% Local Variables: 
%%% mode: latex
%%% TeX-master: "paper"
%%% End: 


\begin{document}

\frame[t]{\titlepage}

\section[Outline]{}
\frame[t]{\tableofcontents}

\section{Introduction}
\subsection{Overview of Topics}

\section{}
\subsection{}

\frame[t] { 
\frametitle{Introduction}
\begin{itemize}
\item Data mining is the search for patterns in large collections of data
\item Pattern recognition is concerned with automatically finding patterns in data
\item Machine learning is pattern recognition with concern for computational tractability
\end{itemize}
}

\frame[t] { 
\frametitle{Example : Handwritten Digit Identification}
\begin{figure}[htbp]
\begin{center}
\includegraphics{"../prmlfigs-pdf-recolored/Figure1_1"}\caption{Hand written digits from the USPS}
\label{fig:1_1}
\end{center}
\end{figure}
}

\frame[t] { 
\frametitle{Approaches to Identifying Digits}
Goal
\begin{itemize}
\item Build a machine that can identify digits automatically
\end{itemize}
Approaches
\begin{itemize}
\item Hand craft a set of rules that separate each digit from the next
\item Set of rules invariably grows large and unwieldy and requires many ``exceptions''
\item ``Learn'' a set of models for each digit automatically from labeled training data.
\end{itemize}
Formalism
\begin{itemize}
\item Each digit is 28x28 pixel image
\item Vectorized into a 784 entry vector $\x$ 
\end{itemize}
}

\frame[t] { 
\frametitle{Machine learning approach}
Recipe
\begin{itemize}
\item Obtain a of $N$ digits $\{\x_1, \ldots, \x_N\}$ called the {\em training set}.
\item Label (by hand) the training set to produce a label or ``target'' $\t$ for each digit image $\x$
\item Learn a function $\y(\x)$ which takes an image $\x$ as input and returns an output in the same ``format'' as the target vector.
\end{itemize}
Terminology
\begin{itemize}
\item The process of determining the precise shape of the function $\y$ is known as the ``training'' or ``learning'' phase.
\item After training, the model (function $\y$) can be used to figure out what digit unseen images might be of.  The set comprised of such data is called the ``test set''
\item If the trained model does a good job of labeling
\end{itemize}




}


\frame[t] { 
\frametitle{}
\begin{figure}[htbp]
\begin{center}
\includegraphics{"../prmlfigs-pdf-recolored/Figure1_2"}\caption{Chapter 1, Figure 2}
\label{fig:1_2}
\end{center}
\end{figure}
}


\frame[t] { 
\frametitle{}
\begin{figure}[htbp]
\begin{center}
\includegraphics{"../prmlfigs-pdf-recolored/Figure1_3"}\caption{Chapter 1, Figure 3}
\label{fig:1_3}
\end{center}
\end{figure}
}


\frame[t] { 
\frametitle{}
\begin{figure}[htbp]
\begin{center}
\includegraphics{"../prmlfigs-pdf-recolored/Figure1_4a"}\caption{Chapter 1, Figure 4a}
\label{fig:1_4a}
\end{center}
\end{figure}
}


\frame[t] { 
\frametitle{}
\begin{figure}[htbp]
\begin{center}
\includegraphics{"../prmlfigs-pdf-recolored/Figure1_4b"}\caption{Chapter 1, Figure 4b}
\label{fig:1_4b}
\end{center}
\end{figure}
}


\frame[t] { 
\frametitle{}
\begin{figure}[htbp]
\begin{center}
\includegraphics{"../prmlfigs-pdf-recolored/Figure1_4c"}\caption{Chapter 1, Figure 4c}
\label{fig:1_4c}
\end{center}
\end{figure}
}


\frame[t] { 
\frametitle{}
\begin{figure}[htbp]
\begin{center}
\includegraphics{"../prmlfigs-pdf-recolored/Figure1_4d"}\caption{Chapter 1, Figure 4d}
\label{fig:1_4d}
\end{center}
\end{figure}
}


\frame[t] { 
\frametitle{}
\begin{figure}[htbp]
\begin{center}
\includegraphics{"../prmlfigs-pdf-recolored/Figure1_5"}\caption{Chapter 1, Figure 5}
\label{fig:1_5}
\end{center}
\end{figure}
}


\frame[t] { 
\frametitle{}
\begin{figure}[htbp]
\begin{center}
\includegraphics{"../prmlfigs-pdf-recolored/Figure1_6a"}\caption{Chapter 1, Figure 6a}
\label{fig:1_6a}
\end{center}
\end{figure}
}


\frame[t] { 
\frametitle{}
\begin{figure}[htbp]
\begin{center}
\includegraphics{"../prmlfigs-pdf-recolored/Figure1_6b"}\caption{Chapter 1, Figure 6b}
\label{fig:1_6b}
\end{center}
\end{figure}
}


\frame[t] { 
\frametitle{}
\begin{figure}[htbp]
\begin{center}
\includegraphics{"../prmlfigs-pdf-recolored/Figure1_7a"}\caption{Chapter 1, Figure 7a}
\label{fig:1_7a}
\end{center}
\end{figure}
}


\frame[t] { 
\frametitle{}
\begin{figure}[htbp]
\begin{center}
\includegraphics{"../prmlfigs-pdf-recolored/Figure1_7b"}\caption{Chapter 1, Figure 7b}
\label{fig:1_7b}
\end{center}
\end{figure}
}


\frame[t] { 
\frametitle{}
\begin{figure}[htbp]
\begin{center}
\includegraphics{"../prmlfigs-pdf-recolored/Figure1_8"}\caption{Chapter 1, Figure 8}
\label{fig:1_8}
\end{center}
\end{figure}
}


\frame[t] { 
\frametitle{}
\begin{figure}[htbp]
\begin{center}
\includegraphics{"../prmlfigs-pdf-recolored/Figure1_9"}\caption{Chapter 1, Figure 9}
\label{fig:1_9}
\end{center}
\end{figure}
}


\frame[t] { 
\frametitle{}
\begin{figure}[htbp]
\begin{center}
\includegraphics{"../prmlfigs-pdf-recolored/Figure1_10"}\caption{Chapter 1, Figure 10}
\label{fig:1_10}
\end{center}
\end{figure}
}


\frame[t] { 
\frametitle{}
\begin{figure}[htbp]
\begin{center}
\includegraphics{"../prmlfigs-pdf-recolored/Figure1_11a"}\caption{Chapter 1, Figure 11a}
\label{fig:1_11a}
\end{center}
\end{figure}
}


\frame[t] { 
\frametitle{}
\begin{figure}[htbp]
\begin{center}
\includegraphics{"../prmlfigs-pdf-recolored/Figure1_11b"}\caption{Chapter 1, Figure 11b}
\label{fig:1_11b}
\end{center}
\end{figure}
}


\frame[t] { 
\frametitle{}
\begin{figure}[htbp]
\begin{center}
\includegraphics{"../prmlfigs-pdf-recolored/Figure1_11c"}\caption{Chapter 1, Figure 11c}
\label{fig:1_11c}
\end{center}
\end{figure}
}


\frame[t] { 
\frametitle{}
\begin{figure}[htbp]
\begin{center}
\includegraphics{"../prmlfigs-pdf-recolored/Figure1_11d"}\caption{Chapter 1, Figure 11d}
\label{fig:1_11d}
\end{center}
\end{figure}
}


\frame[t] { 
\frametitle{}
\begin{figure}[htbp]
\begin{center}
\includegraphics{"../prmlfigs-pdf-recolored/Figure1_12"}\caption{Chapter 1, Figure 12}
\label{fig:1_12}
\end{center}
\end{figure}
}


\frame[t] { 
\frametitle{}
\begin{figure}[htbp]
\begin{center}
\includegraphics{"../prmlfigs-pdf-recolored/Figure1_13"}\caption{Chapter 1, Figure 13}
\label{fig:1_13}
\end{center}
\end{figure}
}


\frame[t] { 
\frametitle{}
\begin{figure}[htbp]
\begin{center}
\includegraphics{"../prmlfigs-pdf-recolored/Figure1_14"}\caption{Chapter 1, Figure 14}
\label{fig:1_14}
\end{center}
\end{figure}
}


\frame[t] { 
\frametitle{}
\begin{figure}[htbp]
\begin{center}
\includegraphics{"../prmlfigs-pdf-recolored/Figure1_15"}\caption{Chapter 1, Figure 15}
\label{fig:1_15}
\end{center}
\end{figure}
}


\frame[t] { 
\frametitle{}
\begin{figure}[htbp]
\begin{center}
\includegraphics{"../prmlfigs-pdf-recolored/Figure1_16"}\caption{Chapter 1, Figure 16}
\label{fig:1_16}
\end{center}
\end{figure}
}


\frame[t] { 
\frametitle{}
\begin{figure}[htbp]
\begin{center}
\includegraphics{"../prmlfigs-pdf-recolored/Figure1_17"}\caption{Chapter 1, Figure 17}
\label{fig:1_17}
\end{center}
\end{figure}
}


\frame[t] { 
\frametitle{}
\begin{figure}[htbp]
\begin{center}
\includegraphics{"../prmlfigs-pdf-recolored/Figure1_18"}\caption{Chapter 1, Figure 18}
\label{fig:1_18}
\end{center}
\end{figure}
}


\frame[t] { 
\frametitle{}
\begin{figure}[htbp]
\begin{center}
\includegraphics{"../prmlfigs-pdf-recolored/Figure1_19"}\caption{Chapter 1, Figure 19}
\label{fig:1_19}
\end{center}
\end{figure}
}


\frame[t] { 
\frametitle{}
\begin{figure}[htbp]
\begin{center}
\includegraphics{"../prmlfigs-pdf-recolored/Figure1_20"}\caption{Chapter 1, Figure 20}
\label{fig:1_20}
\end{center}
\end{figure}
}


\frame[t] { 
\frametitle{}
\begin{figure}[htbp]
\begin{center}
\includegraphics{"../prmlfigs-pdf-recolored/Figure1_21a"}\caption{Chapter 1, Figure 21a}
\label{fig:1_21a}
\end{center}
\end{figure}
}


\frame[t] { 
\frametitle{}
\begin{figure}[htbp]
\begin{center}
\includegraphics{"../prmlfigs-pdf-recolored/Figure1_21b"}\caption{Chapter 1, Figure 21b}
\label{fig:1_21b}
\end{center}
\end{figure}
}


\frame[t] { 
\frametitle{}
\begin{figure}[htbp]
\begin{center}
\includegraphics{"../prmlfigs-pdf-recolored/Figure1_21c"}\caption{Chapter 1, Figure 21c}
\label{fig:1_21c}
\end{center}
\end{figure}
}


\frame[t] { 
\frametitle{}
\begin{figure}[htbp]
\begin{center}
\includegraphics{"../prmlfigs-pdf-recolored/Figure1_22"}\caption{Chapter 1, Figure 22}
\label{fig:1_22}
\end{center}
\end{figure}
}


\frame[t] { 
\frametitle{}
\begin{figure}[htbp]
\begin{center}
\includegraphics{"../prmlfigs-pdf-recolored/Figure1_23"}\caption{Chapter 1, Figure 23}
\label{fig:1_23}
\end{center}
\end{figure}
}


\frame[t] { 
\frametitle{}
\begin{figure}[htbp]
\begin{center}
\includegraphics{"../prmlfigs-pdf-recolored/Figure1_24"}\caption{Chapter 1, Figure 24}
\label{fig:1_24}
\end{center}
\end{figure}
}


\frame[t] { 
\frametitle{}
\begin{figure}[htbp]
\begin{center}
\includegraphics{"../prmlfigs-pdf-recolored/Figure1_26"}\caption{Chapter 1, Figure 26}
\label{fig:1_26}
\end{center}
\end{figure}
}


\frame[t] { 
\frametitle{}
\begin{figure}[htbp]
\begin{center}
\includegraphics{"../prmlfigs-pdf-recolored/Figure1_27a"}\caption{Chapter 1, Figure 27a}
\label{fig:1_27a}
\end{center}
\end{figure}
}


\frame[t] { 
\frametitle{}
\begin{figure}[htbp]
\begin{center}
\includegraphics{"../prmlfigs-pdf-recolored/Figure1_27b"}\caption{Chapter 1, Figure 27b}
\label{fig:1_27b}
\end{center}
\end{figure}
}


\frame[t] { 
\frametitle{}
\begin{figure}[htbp]
\begin{center}
\includegraphics{"../prmlfigs-pdf-recolored/Figure1_28"}\caption{Chapter 1, Figure 28}
\label{fig:1_28}
\end{center}
\end{figure}
}


\frame[t] { 
\frametitle{}
\begin{figure}[htbp]
\begin{center}
\includegraphics{"../prmlfigs-pdf-recolored/Figure1_29a"}\caption{Chapter 1, Figure 29a}
\label{fig:1_29a}
\end{center}
\end{figure}
}


\frame[t] { 
\frametitle{}
\begin{figure}[htbp]
\begin{center}
\includegraphics{"../prmlfigs-pdf-recolored/Figure1_29b"}\caption{Chapter 1, Figure 29b}
\label{fig:1_29b}
\end{center}
\end{figure}
}


\frame[t] { 
\frametitle{}
\begin{figure}[htbp]
\begin{center}
\includegraphics{"../prmlfigs-pdf-recolored/Figure1_29c"}\caption{Chapter 1, Figure 29c}
\label{fig:1_29c}
\end{center}
\end{figure}
}


\frame[t] { 
\frametitle{}
\begin{figure}[htbp]
\begin{center}
\includegraphics{"../prmlfigs-pdf-recolored/Figure1_29d"}\caption{Chapter 1, Figure 29d}
\label{fig:1_29d}
\end{center}
\end{figure}
}


\frame[t] { 
\frametitle{}
\begin{figure}[htbp]
\begin{center}
\includegraphics{"../prmlfigs-pdf-recolored/Figure1_30a"}\caption{Chapter 1, Figure 30a}
\label{fig:1_30a}
\end{center}
\end{figure}
}


\frame[t] { 
\frametitle{}
\begin{figure}[htbp]
\begin{center}
\includegraphics{"../prmlfigs-pdf-recolored/Figure1_30b"}\caption{Chapter 1, Figure 30b}
\label{fig:1_30b}
\end{center}
\end{figure}
}


\frame[t] { 
\frametitle{}
\begin{figure}[htbp]
\begin{center}
\includegraphics{"../prmlfigs-pdf-recolored/Figure1_31"}\caption{Chapter 1, Figure 31}
\label{fig:1_31}
\end{center}
\end{figure}
}

\end{document}