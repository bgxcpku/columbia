\documentclass{beamer}

% \usepackage{beamerthemesplit} // Activate for custom appearance

\title{Introduction}
\author{Frank Wood}
\date{\today}

\newcommand{\ubf}{\mathbf{u}}
\newcommand{\xbf}{\mathbf{x}}
\newcommand{\sbf}{\mathbf{s}}
\newcommand{\py}{\mathcal{PY}}
\newcommand{\vbf}{\mathbf{v}}
\newcommand{\Prob}{\mathrm{P}}
\newcommand{\Psmooth}{\Prob_\text{smooth}}
\newcommand{\parent}{\pi}
\newcommand{\suffix}{\sigma}
\newcommand{\UHPYP}{SM}
\newcommand{\PLUMP}{PLUMP}
\newcommand{\Oh}{\mathcal{O}}
\newcommand{\tree}{\mathcal{T}}

% \newcommand{\cusk}{c_{\ubf s k}}
% \newcommand{\cus}{c_{\ubf s \cdot}}
% \newcommand{\cu}{c_{\ubf \cdot \cdot}}
% \newcommand{\tus}{t_{\ubf s}}
% \newcommand{\tu}{t_{\ubf \cdot}}
\newcommand{\cusk}{c_{\ubf s k}}
\newcommand{\cus}{c_{\ubf s}}
\newcommand{\cu}{c_{\ubf \cdot}}
\newcommand{\tus}{t_{\ubf s}}
\newcommand{\tu}{t_{\ubf \cdot}}
\newcommand{\cset}{\{\cusk\}_{s\in \Sigma,k \in \{1,\ldots,t_{\ubf s}\}}}
\newcommand{\tset}{\{\tus\}_{s\in \Sigma}}
\newcommand{\bydef}{\equiv}
\newcommand{\state}{\mathcal{S}_{\xbf}}
\newcommand{\statei}{\mathcal{S}_{\xbf_{1:i}}}
\newcommand{\emptystring}{\varepsilon}
\newcommand{\gcount}{\hat{c}}
\newcommand{\escape}{\mathtt{esc}}

\newcommand{\todo}[1]{\begin{center}\textbf{TODO: } #1 \end{center}}
\newcommand{\figref}[1]{\figurename~\ref{#1}}
\newcommand{\predictive}{\Prob(x_i|\xbf_{1:i-1})}
\newcommand{\ywcomment}[1]{\textbf{#1}}
\newcommand{\jgcomment}[1]{ { \textcolor{red}{#1} } }



\begin{document}

\frame[t]{\titlepage}

\section[Outline]{}
\frame[t]{\tableofcontents}

\section{Introduction}
\subsection{Overview of Topics}

\section{}
\subsection{}

\frame[t] { 
\frametitle{Introduction}
\begin{itemize}
\item Data mining is the search for patterns in large collections of data
\item Pattern recognition is concerned with automatically finding patterns in data
\item Machine learning is pattern recognition with concern for computational tractability
\end{itemize}
}

\frame[t] { 
\frametitle{Example : Handwritten Digit Identification}
\begin{figure}[htbp]
\begin{center}
\includegraphics{"../prmlfigs-pdf-recolored/Figure1_1"}\caption{Hand written digits from the USPS}
\label{fig:1_1}
\end{center}
\end{figure}
}

\frame[t] { 
\frametitle{Approaches to Identifying Digits}
Goal
\begin{itemize}
\item Build a machine that can identify digits automatically
\end{itemize}
Approaches
\begin{itemize}
\item Hand craft a set of rules that separate each digit from the next
\item Set of rules invariably grows large and unwieldy and requires many ``exceptions''
\item ``Learn'' a set of models for each digit automatically from labeled training data.
\end{itemize}
Formalism
\begin{itemize}
\item Each digit is 28x28 pixel image
\item Vectorized into a 784 entry vector $\x$ 
\end{itemize}
}

\frame[t] { 
\frametitle{Machine learning approach}
Recipe
\begin{itemize}
\item Obtain a of $N$ digits $\{\x_1, \ldots, \x_N\}$ called the {\em training set}.
\item Label (by hand) the training set to produce a label or ``target'' $\t$ for each digit image $\x$
\item Learn a function $\y(\x)$ which takes an image $\x$ as input and returns an output in the same ``format'' as the target vector.
\end{itemize}
Terminology
\begin{itemize}
\item The process of determining the precise shape of the function $\y$ is known as the ``training'' or ``learning'' phase.
\item After training, the model (function $\y$) can be used to figure out what digit unseen images might be of.  The set comprised of such data is called the ``test set''
\item If the trained model does a good job of labeling
\end{itemize}




}


\frame[t] { 
\frametitle{}
\begin{figure}[htbp]
\begin{center}
\includegraphics{"../prmlfigs-pdf-recolored/Figure1_2"}\caption{Chapter 1, Figure 2}
\label{fig:1_2}
\end{center}
\end{figure}
}


\frame[t] { 
\frametitle{}
\begin{figure}[htbp]
\begin{center}
\includegraphics{"../prmlfigs-pdf-recolored/Figure1_3"}\caption{Chapter 1, Figure 3}
\label{fig:1_3}
\end{center}
\end{figure}
}


\frame[t] { 
\frametitle{}
\begin{figure}[htbp]
\begin{center}
\includegraphics{"../prmlfigs-pdf-recolored/Figure1_4a"}\caption{Chapter 1, Figure 4a}
\label{fig:1_4a}
\end{center}
\end{figure}
}


\frame[t] { 
\frametitle{}
\begin{figure}[htbp]
\begin{center}
\includegraphics{"../prmlfigs-pdf-recolored/Figure1_4b"}\caption{Chapter 1, Figure 4b}
\label{fig:1_4b}
\end{center}
\end{figure}
}


\frame[t] { 
\frametitle{}
\begin{figure}[htbp]
\begin{center}
\includegraphics{"../prmlfigs-pdf-recolored/Figure1_4c"}\caption{Chapter 1, Figure 4c}
\label{fig:1_4c}
\end{center}
\end{figure}
}


\frame[t] { 
\frametitle{}
\begin{figure}[htbp]
\begin{center}
\includegraphics{"../prmlfigs-pdf-recolored/Figure1_4d"}\caption{Chapter 1, Figure 4d}
\label{fig:1_4d}
\end{center}
\end{figure}
}


\frame[t] { 
\frametitle{}
\begin{figure}[htbp]
\begin{center}
\includegraphics{"../prmlfigs-pdf-recolored/Figure1_5"}\caption{Chapter 1, Figure 5}
\label{fig:1_5}
\end{center}
\end{figure}
}


\frame[t] { 
\frametitle{}
\begin{figure}[htbp]
\begin{center}
\includegraphics{"../prmlfigs-pdf-recolored/Figure1_6a"}\caption{Chapter 1, Figure 6a}
\label{fig:1_6a}
\end{center}
\end{figure}
}


\frame[t] { 
\frametitle{}
\begin{figure}[htbp]
\begin{center}
\includegraphics{"../prmlfigs-pdf-recolored/Figure1_6b"}\caption{Chapter 1, Figure 6b}
\label{fig:1_6b}
\end{center}
\end{figure}
}


\frame[t] { 
\frametitle{}
\begin{figure}[htbp]
\begin{center}
\includegraphics{"../prmlfigs-pdf-recolored/Figure1_7a"}\caption{Chapter 1, Figure 7a}
\label{fig:1_7a}
\end{center}
\end{figure}
}


\frame[t] { 
\frametitle{}
\begin{figure}[htbp]
\begin{center}
\includegraphics{"../prmlfigs-pdf-recolored/Figure1_7b"}\caption{Chapter 1, Figure 7b}
\label{fig:1_7b}
\end{center}
\end{figure}
}


\frame[t] { 
\frametitle{}
\begin{figure}[htbp]
\begin{center}
\includegraphics{"../prmlfigs-pdf-recolored/Figure1_8"}\caption{Chapter 1, Figure 8}
\label{fig:1_8}
\end{center}
\end{figure}
}


\frame[t] { 
\frametitle{}
\begin{figure}[htbp]
\begin{center}
\includegraphics{"../prmlfigs-pdf-recolored/Figure1_9"}\caption{Chapter 1, Figure 9}
\label{fig:1_9}
\end{center}
\end{figure}
}


\frame[t] { 
\frametitle{}
\begin{figure}[htbp]
\begin{center}
\includegraphics{"../prmlfigs-pdf-recolored/Figure1_10"}\caption{Chapter 1, Figure 10}
\label{fig:1_10}
\end{center}
\end{figure}
}


\frame[t] { 
\frametitle{}
\begin{figure}[htbp]
\begin{center}
\includegraphics{"../prmlfigs-pdf-recolored/Figure1_11a"}\caption{Chapter 1, Figure 11a}
\label{fig:1_11a}
\end{center}
\end{figure}
}


\frame[t] { 
\frametitle{}
\begin{figure}[htbp]
\begin{center}
\includegraphics{"../prmlfigs-pdf-recolored/Figure1_11b"}\caption{Chapter 1, Figure 11b}
\label{fig:1_11b}
\end{center}
\end{figure}
}


\frame[t] { 
\frametitle{}
\begin{figure}[htbp]
\begin{center}
\includegraphics{"../prmlfigs-pdf-recolored/Figure1_11c"}\caption{Chapter 1, Figure 11c}
\label{fig:1_11c}
\end{center}
\end{figure}
}


\frame[t] { 
\frametitle{}
\begin{figure}[htbp]
\begin{center}
\includegraphics{"../prmlfigs-pdf-recolored/Figure1_11d"}\caption{Chapter 1, Figure 11d}
\label{fig:1_11d}
\end{center}
\end{figure}
}


\frame[t] { 
\frametitle{}
\begin{figure}[htbp]
\begin{center}
\includegraphics{"../prmlfigs-pdf-recolored/Figure1_12"}\caption{Chapter 1, Figure 12}
\label{fig:1_12}
\end{center}
\end{figure}
}


\frame[t] { 
\frametitle{}
\begin{figure}[htbp]
\begin{center}
\includegraphics{"../prmlfigs-pdf-recolored/Figure1_13"}\caption{Chapter 1, Figure 13}
\label{fig:1_13}
\end{center}
\end{figure}
}


\frame[t] { 
\frametitle{}
\begin{figure}[htbp]
\begin{center}
\includegraphics{"../prmlfigs-pdf-recolored/Figure1_14"}\caption{Chapter 1, Figure 14}
\label{fig:1_14}
\end{center}
\end{figure}
}


\frame[t] { 
\frametitle{}
\begin{figure}[htbp]
\begin{center}
\includegraphics{"../prmlfigs-pdf-recolored/Figure1_15"}\caption{Chapter 1, Figure 15}
\label{fig:1_15}
\end{center}
\end{figure}
}


\frame[t] { 
\frametitle{}
\begin{figure}[htbp]
\begin{center}
\includegraphics{"../prmlfigs-pdf-recolored/Figure1_16"}\caption{Chapter 1, Figure 16}
\label{fig:1_16}
\end{center}
\end{figure}
}


\frame[t] { 
\frametitle{}
\begin{figure}[htbp]
\begin{center}
\includegraphics{"../prmlfigs-pdf-recolored/Figure1_17"}\caption{Chapter 1, Figure 17}
\label{fig:1_17}
\end{center}
\end{figure}
}


\frame[t] { 
\frametitle{}
\begin{figure}[htbp]
\begin{center}
\includegraphics{"../prmlfigs-pdf-recolored/Figure1_18"}\caption{Chapter 1, Figure 18}
\label{fig:1_18}
\end{center}
\end{figure}
}


\frame[t] { 
\frametitle{}
\begin{figure}[htbp]
\begin{center}
\includegraphics{"../prmlfigs-pdf-recolored/Figure1_19"}\caption{Chapter 1, Figure 19}
\label{fig:1_19}
\end{center}
\end{figure}
}


\frame[t] { 
\frametitle{}
\begin{figure}[htbp]
\begin{center}
\includegraphics{"../prmlfigs-pdf-recolored/Figure1_20"}\caption{Chapter 1, Figure 20}
\label{fig:1_20}
\end{center}
\end{figure}
}


\frame[t] { 
\frametitle{}
\begin{figure}[htbp]
\begin{center}
\includegraphics{"../prmlfigs-pdf-recolored/Figure1_21a"}\caption{Chapter 1, Figure 21a}
\label{fig:1_21a}
\end{center}
\end{figure}
}


\frame[t] { 
\frametitle{}
\begin{figure}[htbp]
\begin{center}
\includegraphics{"../prmlfigs-pdf-recolored/Figure1_21b"}\caption{Chapter 1, Figure 21b}
\label{fig:1_21b}
\end{center}
\end{figure}
}


\frame[t] { 
\frametitle{}
\begin{figure}[htbp]
\begin{center}
\includegraphics{"../prmlfigs-pdf-recolored/Figure1_21c"}\caption{Chapter 1, Figure 21c}
\label{fig:1_21c}
\end{center}
\end{figure}
}


\frame[t] { 
\frametitle{}
\begin{figure}[htbp]
\begin{center}
\includegraphics{"../prmlfigs-pdf-recolored/Figure1_22"}\caption{Chapter 1, Figure 22}
\label{fig:1_22}
\end{center}
\end{figure}
}


\frame[t] { 
\frametitle{}
\begin{figure}[htbp]
\begin{center}
\includegraphics{"../prmlfigs-pdf-recolored/Figure1_23"}\caption{Chapter 1, Figure 23}
\label{fig:1_23}
\end{center}
\end{figure}
}


\frame[t] { 
\frametitle{}
\begin{figure}[htbp]
\begin{center}
\includegraphics{"../prmlfigs-pdf-recolored/Figure1_24"}\caption{Chapter 1, Figure 24}
\label{fig:1_24}
\end{center}
\end{figure}
}


\frame[t] { 
\frametitle{}
\begin{figure}[htbp]
\begin{center}
\includegraphics{"../prmlfigs-pdf-recolored/Figure1_26"}\caption{Chapter 1, Figure 26}
\label{fig:1_26}
\end{center}
\end{figure}
}


\frame[t] { 
\frametitle{}
\begin{figure}[htbp]
\begin{center}
\includegraphics{"../prmlfigs-pdf-recolored/Figure1_27a"}\caption{Chapter 1, Figure 27a}
\label{fig:1_27a}
\end{center}
\end{figure}
}


\frame[t] { 
\frametitle{}
\begin{figure}[htbp]
\begin{center}
\includegraphics{"../prmlfigs-pdf-recolored/Figure1_27b"}\caption{Chapter 1, Figure 27b}
\label{fig:1_27b}
\end{center}
\end{figure}
}


\frame[t] { 
\frametitle{}
\begin{figure}[htbp]
\begin{center}
\includegraphics{"../prmlfigs-pdf-recolored/Figure1_28"}\caption{Chapter 1, Figure 28}
\label{fig:1_28}
\end{center}
\end{figure}
}


\frame[t] { 
\frametitle{}
\begin{figure}[htbp]
\begin{center}
\includegraphics{"../prmlfigs-pdf-recolored/Figure1_29a"}\caption{Chapter 1, Figure 29a}
\label{fig:1_29a}
\end{center}
\end{figure}
}


\frame[t] { 
\frametitle{}
\begin{figure}[htbp]
\begin{center}
\includegraphics{"../prmlfigs-pdf-recolored/Figure1_29b"}\caption{Chapter 1, Figure 29b}
\label{fig:1_29b}
\end{center}
\end{figure}
}


\frame[t] { 
\frametitle{}
\begin{figure}[htbp]
\begin{center}
\includegraphics{"../prmlfigs-pdf-recolored/Figure1_29c"}\caption{Chapter 1, Figure 29c}
\label{fig:1_29c}
\end{center}
\end{figure}
}


\frame[t] { 
\frametitle{}
\begin{figure}[htbp]
\begin{center}
\includegraphics{"../prmlfigs-pdf-recolored/Figure1_29d"}\caption{Chapter 1, Figure 29d}
\label{fig:1_29d}
\end{center}
\end{figure}
}


\frame[t] { 
\frametitle{}
\begin{figure}[htbp]
\begin{center}
\includegraphics{"../prmlfigs-pdf-recolored/Figure1_30a"}\caption{Chapter 1, Figure 30a}
\label{fig:1_30a}
\end{center}
\end{figure}
}


\frame[t] { 
\frametitle{}
\begin{figure}[htbp]
\begin{center}
\includegraphics{"../prmlfigs-pdf-recolored/Figure1_30b"}\caption{Chapter 1, Figure 30b}
\label{fig:1_30b}
\end{center}
\end{figure}
}


\frame[t] { 
\frametitle{}
\begin{figure}[htbp]
\begin{center}
\includegraphics{"../prmlfigs-pdf-recolored/Figure1_31"}\caption{Chapter 1, Figure 31}
\label{fig:1_31}
\end{center}
\end{figure}
}

\end{document}