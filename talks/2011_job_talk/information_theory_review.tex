\documentclass[16pt]{beamer}
\usepackage{algorithm}
\usepackage{algorithmic}
\usepackage[numbers]{natbib}
\usepackage{pgf,pgfarrows,pgfnodes}
% \usepackage{beamerthemesplit} // Activate for custom appearance

\title{Information Theory Review}
\author{Frank Wood}
\date{\today}



\usetheme{Darmstadt}
%\usetheme{Copenhagen}
%\usefonttheme[onlylarge]{structurebold}
%\setbeamerfont*{frametitle}{size=\normalsize,series=\bfseries}
%\setbeamertemplate{navigation symbols}{}

\usepackage{amsfonts}
\usepackage{amsmath}
%\newcommand{\argmax}{\operatornamewithlimits{argmax}}
\def\newblock{\hskip .11em plus .33em minus .07em}
% Setup TikZ

\newcommand{\ubf}{\mathbf{u}}
\newcommand{\xbf}{\mathbf{x}}
\newcommand{\sbf}{\mathbf{s}}
\newcommand{\py}{\mathcal{PY}}
\newcommand{\vbf}{\mathbf{v}}
\newcommand{\Prob}{\mathrm{P}}
\newcommand{\Psmooth}{\Prob_\text{smooth}}
\newcommand{\parent}{\pi}
\newcommand{\suffix}{\sigma}
\newcommand{\UHPYP}{SM}
\newcommand{\PLUMP}{PLUMP}
\newcommand{\Oh}{\mathcal{O}}
\newcommand{\tree}{\mathcal{T}}

% \newcommand{\cusk}{c_{\ubf s k}}
% \newcommand{\cus}{c_{\ubf s \cdot}}
% \newcommand{\cu}{c_{\ubf \cdot \cdot}}
% \newcommand{\tus}{t_{\ubf s}}
% \newcommand{\tu}{t_{\ubf \cdot}}
\newcommand{\cusk}{c_{\ubf s k}}
\newcommand{\cus}{c_{\ubf s}}
\newcommand{\cu}{c_{\ubf \cdot}}
\newcommand{\tus}{t_{\ubf s}}
\newcommand{\tu}{t_{\ubf \cdot}}
\newcommand{\cset}{\{\cusk\}_{s\in \Sigma,k \in \{1,\ldots,t_{\ubf s}\}}}
\newcommand{\tset}{\{\tus\}_{s\in \Sigma}}
\newcommand{\bydef}{\equiv}
\newcommand{\state}{\mathcal{S}_{\xbf}}
\newcommand{\statei}{\mathcal{S}_{\xbf_{1:i}}}
\newcommand{\emptystring}{\varepsilon}
\newcommand{\gcount}{\hat{c}}
\newcommand{\escape}{\mathtt{esc}}

\newcommand{\todo}[1]{\begin{center}\textbf{TODO: } #1 \end{center}}
\newcommand{\figref}[1]{\figurename~\ref{#1}}
\newcommand{\predictive}{\Prob(x_i|\xbf_{1:i-1})}
\newcommand{\ywcomment}[1]{\textbf{#1}}
\newcommand{\jgcomment}[1]{ { \textcolor{red}{#1} } }




\title[Information Theory] 
{
	Information Theory Review
}

\author[http://www.stat.columbia.edu/$\sim$fwood]
{
  Frank~Wood \\ 
}

\institute[Columbia University]
{
  %\inst{1}%
  Columbia University
}

\begin{document}
\frame{\titlepage}
\section{Compression}
\subsection{Information theory review}
\frame[t]{
\begin{block}{Shannon's Entropy \cite{Shannon1948,MacKay2003}}
With $X$ a R.V. taking values $x_i \in \Sigma$ with probability $P(X=x_i)$, the entropy of the ``ensemble'' $\mathcal{X} =\{X,\Sigma,P\}$ is
\[H(\mathcal{X}) = \sum_i P(X=x_i) \log_2\left(\frac{1}{P(X=x_i)}\right)\]
\begin{itemize}
\item Units of ``bits''
\item Maximized when $P(X = x_i) = \frac{1}{|\Sigma|}$
\item Lower bound on average number of bits required to losslessly transmit message consisting of symbols $x_i \sim P$ over a noiseless channel.
\end{itemize}

\end{block}
}

\frame[t]{
\begin{block}{Symbol code \cite{Shannon1948,MacKay2003}}
A binary code C is a mapping
\[C : \Sigma \rightarrow \{0,1\}^+\]
with expected code length
\[ L(C,\mathcal{X}) = \sum_{x_i\in\Sigma} P(X = x_i)\ell(x_i)\] 
where $\ell(x_i)$ is length of the binary string assigned to $x_i$.
\end{block}

}

\frame[t]{
\begin{block}{Source coding theorem: bound on prefix  code lengths }
For an ensemble $\mathcal{X}$ there exists a prefix (uniquely identifiable)\footnote{``A code $C$ is uniquely decodeable if, under the extended code $C^+$, no two distinct strings have the same encoding, i.e.''\cite{MacKay2003} \[\forall \x,\y \in \Sigma^+, \x \neq \y \implies C^+(\x) \neq C^+(\y)\] where \[C^+(\x) = C(x_1)C(x_2)\ldots\]\vspace{-.25cm}} code $C$ with expected length satisfying
\[ H(\mathcal{X}) \leq L(C,\mathcal{X}) \leq  H(\mathcal{X}) +1 \]
\vspace{.5cm}
\end{block}
}

\subsection{A Huffman (''optimal'') code for unigram English character probabilities}
\frame[t]{
\begin{figure}[t]
    \begin{center}
        \includegraphics[width=.6\columnwidth]{huffman}
    \end{center}
    \label{fig:huffman}
\end{figure}
Huffman codes are constructed by ``bottom-up'' agglomerative merging of lowest probability symbols. \\ {\tiny Figure from \cite{MacKay2003}}.
}

\subsection{Doing better than Huffman coding}
\frame[t]{
\begin{block}{Drawbacks of Huffman codes}
Only optimal if source is i.i.d. and single symbol transmitted
\begin{itemize}
\item $H(\mathcal{X}) \leq L(C,\mathcal{X}) \leq  H(\mathcal{X}) +1$ bound is {\em per-symbol}
\item $H(\mathcal{X}_2|\mathcal{X}_1) \leq H(\mathcal{X}_2)$, information always reduces entropy.
\item Entire source must be processed to estimate the optimal codebook 
\item Codebook must be transmitted before data -- often highly redundant
\end{itemize}
\end{block}
\begin{block}{Fixes}
\begin{itemize}
\item Code {\em blocks} or ``words'' $\x = [x_1,x_2,\ldots,x_N] \in \Sigma^N$ of the input (trade-off -- requires more complex model)
\item Incrementally build codebook (inefficient, bottom-up)
\item Use {\em arithmetic coding} instead. \end{itemize}
\end{block}
}

\subsection{Compressing `$\mbox{bbba}\square$' using arithmetic coding}
\frame[t]{
Arithmetic coding can be seen as using fractions of bits to encode a symbol.
\begin{figure}[t]
    \begin{center}
        \includegraphics[width=.8\columnwidth]{predictive_probs}\footnote{From \cite{MacKay2003}}
    \end{center}
    \label{fig:predictive_probs}
\end{figure}
Example predictive probabilities over English characters.
}
\frame[t]{

%\begin{pgfpicture}{0cm}{0cm}{5cm}{5cm} 
% (0cm,0cm) is the lower left corner, 
% (5cm,2cm) is the upper right corner. 
%\pgfrect[stroke]{\pgfpoint{0cm}{0cm}}{\pgfpoint{2cm}{10pt}} 
% Paint a rectangle (stroke it, do not fill it)
% The lower left corner is at (0cm,0cm) 
% The rectangle is 2cm wide and 10pt high. 
%\pgfcircle[fill]{\pgfpoint{3cm}{1cm}}{10pt} 
% Paint a filled circle % The center is at (3cm,1cm) % The radius is 10pt
% \end{pgfpicture}

\begin{figure}[t]
    \begin{center}
        \includegraphics[width=.8\columnwidth]{arithmetic_coding}
        %\pgfsetcolor{fill}{white}
        \pgfrect[fill]{\pgfpoint{-1cm}{5.5cm}}{\pgfpoint{2cm}{2cm}} 
    \end{center}
    \label{fig: arithmetic_coding}
\end{figure}
Schematic of arithmetic coding {\tiny Figure from \cite{MacKay2003}}.


}

\frame[t]{
\begin{block}{Average code length for arithmetic coding \cite{sayood2000}}
If we redefine $X$ to be a R.V. taking values $\x_i \in \Sigma^+$ with probability $P(X=\x_i)$, and consider the ensemble $\mathcal{X} =\{X,\Sigma^+,P\}$ an arithmetic code $C$ has expected length $L(C,\mathcal{X})$ satisfying
\[ H(\mathcal{X}) \leq L(C,\mathcal{X}) \leq  H(\mathcal{X}) +2 \]
\vspace{.5cm}
\end{block}
}

\subsection{Recipe}
\frame[t]{
\begin{block}{Streaming compressor, encoder}
Repeat
\begin{itemize}
\item Compute cumulative predictive distribution $F(x_i|x_{1:(i-1)})$
\item Send CDF boundary for $x_i$ to arithmetic encoder
\item Update model to include $x_i$ in {\em context} $x_{1:(i-1)}$
\end{itemize}
\end{block}


\begin{block}{Streaming compressor, decoder}
Repeat
\begin{itemize}
\item Compute cumulative predictive distribution $F(x_i|x_{1:(i-1)})$
\item Process bits until CDF interval distinct
\item Conclude $x_i$ sent from encoder
\item Update model to include $x_i$ in {\em context} $x_{1:(i-1)}$
\end{itemize}

\end{block}
}


\subsection{Bibliography}
\bibliographystyle{plainnat}
	\begin{frame}[t,allowframebreaks]{}

\bibliography{../../papers/uber}
\end{frame}
%\begin{alertblock}{Nature of Problems}
%Computational
%\end{alertblock}

\end{document}

