\label{powerlaw}

\section{Power Law Processes}

\subsection{Power Law Distribution}

\frame[t] {
\frametitle{Definition}
A power law distribution is a distribution whose tail falls off as
\[ P(X=x) \propto x^{-\alpha} \]
For some $\alpha > 1$.  The simplest such distribution is the zeta distribution on natural numbers:
\[ f_\alpha(k) = \frac{k^{-\alpha}}{\zeta(\alpha)} \]
and the Pareto distribution on the real line (with small-scale cutoff $x_{min} > 0$):
\[ f_\alpha(x) = \frac{\alpha-1}{x_{min}} \left(\frac{x}{x_{min}}\right)^{-\alpha} \]
}

\frame[t] {
\frametitle{Scale Free Property}
A power law distribution on the real line is {\em scale free} in the sense that, for all scales $k > 0$
\[  f_{\alpha}(kx) \propto f_{\alpha}(x) \]
Where the proportionality constant depends only on $k$.
}

\subsection{Examples}

\frame[t] {
\frametitle{Power Laws in Nature}
Power law relationships appear frequently in complex natural phenomena:
\begin{itemize}
\item Words in natural language, arranged by frequency \cite{Zipf1965}.
\item Cities arranged by population \cite{Blank2000}.
\item Earthquakes arranged by magnitude \cite{Gutenberg1955}.
\item Power spectra of natural images \cite{Ruderman1994}.
\end{itemize}
}

\subsection{Pitman-Yor Process}

\frame[t] {
\frametitle{Definition of the PYP}
}

\frame[t] {
\frametitle{Chinese Restaurant Representation}
}