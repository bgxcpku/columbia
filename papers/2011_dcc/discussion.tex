% !TEX root = deplump.tex
\section{Discussion}
\label{sec:discussion}

In this paper we demonstrate successful deployment of deplump on streaming corpora and detail the algorithm in detail.  We show the approximations introduced to render the algorithm tractable in a streaming setting have almost no effect on the compression performance in comparison to batch deplump.  Results indicate that streaming deplump continues to improve as the stream length increases up to 1,000Mb.  Furthermore, we show that a reasonable bound on the total count in each suffix tree node has no effect on compression performance. We demonstrate that the depth of the tree can be a fixed low constant (10 - 32) without adverse effect on performance and we note that the algorithm performs better when the number of nodes in the tree ($L$) is set high ($10^6$ or $10^7$).

There are both theoretical and empirical questions to pursue in future work.  Of theoretical interest is the exploration of more expressive graphical models. Useful extensions might be made either by introducing sharing structures other than suffix trees, or by considering latent distributed (feature) representations of the stochastic generative process.  We would also like a more explicit mathematical characterization of how the approximations we make effect the inference procedure.  Of empirical interest is experimentation with other media types.  We would also like to investigate the possibility of explicit model mixing to capture other aspects of the conditional distributions.  