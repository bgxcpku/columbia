% !TEX root = deplump.tex
\section{Discussion}
\label{sec:discussion}

We demonstrate successful deployment of deplump on streaming corpora.  We show the approximations introduced to render the algorithm useful in a streaming setting have almost no effect on the compression performance in comparison to batch deplump.   Results indicate that streaming deplump continues to improve as the stream length increases up to 1,000Mb.  Experimentation shows a reasonable bound on the total count in each suffix tree node has no effect on compression performance. We demonstrate that the depth of the tree can be a fixed low constant (10 - 32) without adverse effect on performance and we note that the algorithm performs better when the number of nodes in the tree ($L$) is set high ($10^6$ or $10^7$).

There are both theoretical and empirical questions to pursue in future work.  On the theoretical side we would like to consider more expressive graphical models, either by introducing sharing structures other than suffix trees, or by considering latent representations of the stochastic generative processes.  We would also like a more explicit characterization of how the approximations we make effect the inference procedure.  Empirically we would like to experiment with other media types and explicit model mixing to capture other aspects of the conditional distributions.  
