\section{Bayesian PDFA Inference}

The outline of this section is as follows.  We define a prior distribution over PDFAs with a fixed number of states and show that many of the parameters can be integrated out.  We derive a Metropolis-Hastings sampler for posterior inference in the finite model, and show that many of the elements of the transition matrix can be ignored without affecting the correctness of sampling.  We then let the number of states go to infinity and show that the limit is well defined.  This is the probabilistic deterministic infinite automaton (PDIA).  Inference in the PDIA carries over from the finite case in a natural way.

\subsection{A Prior over PDFAs}

We assume that the states $Q$, alphabet $\Sigma$ and initial state $q_0$ are known, and define a prior over the transition function $\delta$ and emission probabilities $\pi$.  In the finite case $\delta$ and $\pi$ can be thought of as matrices, with columns indexing elements of $\Sigma$ and rows indexing elements of $Q$.  For each column $j$ of the transition matrix $\delta$, we sample the rows i.i.d. from a discrete distribution $\bphi_j$ over $Q$, that is, $\delta \sim [\bphi_1\ldots\bphi_{|\Sigma|}]$.  The $\bphi_j$ themselves are sampled i.i.d. from a Dirichlet prior with parameters $\alpha\bmu$, where $\alpha$ is a concentration and $\bmu$ is itself drawn from a uniform Dirichlet distribution, with $\gamma$ total pseudocounts.  This hierarchical Dirichlet construction allows elements of $\delta$ to be coupled together in a way that remains well-behaved in the limit of infinite states.  We also place a uniform Dirichlet prior over the per-state emission probabilities $\bpi_i$ with $\beta$ total pseudocounts.  Formally:

\begin{eqnarray}
\bmu|\gamma,|Q| & \sim & \mathrm{Dir}\left(\gamma/|Q|,\ldots,\gamma/|Q|\right) \\
\bphi_{j}|\alpha,\bmu  & \sim & \mathrm{Dir}(\alpha\bmu) \\
\bpi_{i}|\beta,|\Sigma| & \sim & \mathrm{Dir}(\beta/|\Sigma|,\ldots,\beta/|\Sigma|)\\
\delta_{ij} & \sim & \bphi_{j}
\end{eqnarray}

where $i$ and $j$ go from 1 to $|Q|$ and $|\Sigma|$ respectively.  From this we generate a sequence of $T$ symbols:

\begin{eqnarray*}
\q_0 & = & q_0 \\
x_0 & \sim & \bpi_0 \\
\q_t & = & \delta(\q_{t-1},x_{t-1}) \\
x_t & \sim & \bpi_t
\end{eqnarray*}

We choose this particular inductive bias, with transitions tied together within a column of $\delta$, because the most recent symbol ought to be informative about what the next state is.  If we instead had a single Dirichlet prior over all elements of $\delta$, transitions to a few states would be highly likely no matter the context and those states would dominate the behavior of the automata.  If we tied together rows of $\delta$ instead of columns, being in a particular state would tell us more about the sequence of states we came from than the symbols that got us there.  As we would like states to be good statistics of the past for predicting the future, a bias that depends on the most recent symbol is preferable.

Given data, the likelihood is

\[ p(x_{0:T}|\delta,\pi) = \pi(\q_0,x_0)\prod_{t=1}^T \pi(\q_t,x_t) \]

We can marginalize out $\pi$ and express the likelihood in a form that depends only on the counts of symbols emitted from each state.  Define the count matrix $c$ for the sequence $x_{0:T}$ and transition matrix $\delta$ as $c_{ij} = \displaystyle\sum_{t=0}^T I_{ij}(\q_t,x_t)$, where $I_{ij}(\q_t,x_t)$ is an indicator function that is 1 if $\q_t = q_i$ and $x_t = \sigma_j$, 0 otherwise. This matrix gives the number of times each symbol is emitted from each state.  Thanks to multinomial-Dirichlet conjugacy we can integrate out $\pi$ and express the probability of a sequence given the transition function $\delta$ solely in terms of $c$ and $\beta$:

\begin{eqnarray}
 p(x_{0:T}|\delta,\beta) & = & \int p(x_{0:T}|\pi,\delta) p(\pi|\beta) d\pi \\
 & = &  \prod_{i=1}^{|Q|} \int \frac{\Gamma(\beta)}{\Gamma(\frac{\beta}{|\Sigma|})^{|\Sigma|}} \pi_{i1}^{\frac{\beta}{|\Sigma|}+c_{i1}-1} \pi_{i2}^{\frac{\beta}{|\Sigma|}+c_{i2}-1} \ldots \pi_{i|\Sigma|}^{\frac{\beta}{|\Sigma|}+c_{i|\Sigma|}-1} d\bpi_i\\
 & = & \prod_{i=1}^{|Q|} \frac{\Gamma(\beta)}{\Gamma(\frac{\beta}{|\Sigma|})^{|\Sigma|}} \frac{\prod_{j=1}^{|\Sigma|}\Gamma(\frac{\beta}{|\Sigma|} + c_{ij})}{\Gamma(\beta + \sum_{j=1}^{|\Sigma|} c_{ij})}
 \end{eqnarray}
 
 The dependence of $c$ on $\delta$ is only through $\q_{0:T}$, and changing a single element of $\delta$ can have a complex effect on $c$.  By changing a single transition $\delta_{ij}$, all states downstream of the first time $q_i$ emits $\sigma_j$ are affected, and some elements of $c$ that were zero may become nonzero, and vice versa.
 
 \subsection{Posterior Inference in the Finite Model}
 
 \subsection{The Probabilistic Deterministic Infinite Automaton}
 
 For the transition matrix, the two-stage Dirichlet prior has the effect of tying together distributions over different columns.  This means that the state/symbol transitions are more similar for transitions from the same symbol, but are still shared across completely different state/symbol pairs.  As $|Q|\rightarrow\infty$, Equations 1 and 2 (*replace this with LaTeX equation titles*) have a well-defined limit as long as the number of observations is finite.  This is the Hierarchical Dirichlet Process.  In the case of posterior inference of $\delta$, when only a finite number of state/symbol pairs are visited by the data, we may integrate out all other elements of $\delta$, and the limit in the case of infinite states is well-defined.
 
We can sample incrementally from the joint distribution over $x_{0:T}$ and $\delta$ when $\phi_j$, $\mu$ and $\pi_i$ are integrated out in the $|Q|\rightarrow\infty$ limit.  From the start state $q_0$, we sample a symbol $s_{j_0}$ uniformly and assign $\delta_{0j_0}$ to a new state.  If $q^t = q_i$ then $x_t$ has the probability

 \[P(x_t=s_j|q_i,x_{0:t-1},q^{0:t-1}) = \frac{c_{ij}+\frac{\beta}{|\Sigma|}}{c_{i\cdot} + \beta}\]
 
 where $c_{ij}$ is the number of times so far $s_j$ was emitted from $q_i$ and $c_{i\cdot}$ is the total number of times $q_i$ has been visited so far.  
 
If the state/symbol pair $(q_i,s_j)$ has not been visited before, we have to sample $\delta_{ij}$.  The two-stage generative procedure for elements of $\delta$ means that we have to keep track of counts at two levels.  Each $\delta_{ij}$ belongs to a cluster $v_{kj}$ that contains other $\delta_{i'j}$, while each $v_{kj}$ belongs to a top-level cluster $w_{l}$ that has elements across all $j$.  Each top level cluster has one $q \in Q$ assigned to it, and $\delta_{ij}$ is equal to that $q$ in the top cluster that the cluster with $\delta_{ij}$ belongs to (*might want to make this part clearer...add a figure*).  Let $\delta^t$ denote the elements of $\delta$ that have been visited at time $t$.  as follows:
 
%\[P(\delta_{ij} = k|\delta^t) \propto \begin{cases} & if $k \leq |\delta^t|$ \cr  & if $k > |\delta^t|$ \end{cases}\]
 
 This process for sampling from an HDP when $\mu$ and $\phi_j$ are integrated out is known as the {\em Chinese Restaurant Franchise Process}.