% Essential Formatting

\documentclass[12pt]{article}
\usepackage{epsfig,amsmath,amsthm,amssymb}
%\usepackage[questions, answersheet]{urmathtest}[2001/05/12]
%\usepackage[answersheet]{urmathtest}[2001/05/12]
\usepackage[answers]{urmathtest}[2001/05/12]


% For use with pdflatex
% \pdfpagewidth\paperwidth
% \pdfpageheight\paperheight

% Basic User Defs

\def\ds{\displaystyle}

\newcommand{\ansbox}[1]
{\work{
  \pos\hfill \framebox[#1][l]{ANSWER:\rule[-.3in]{0in}{.7in}}
}{}}

\newcommand{\ansrectangle}
{\work{
  \pos\hfill \framebox[6in][l]{ANSWER:\rule[-.3in]{0in}{.7in}}
}{}}


% Beginning of the Document

\begin{document}
\examtitle{LINEAR REGRESSION MODELS W4315}{HOMEWORK 1}{09/16/2009}
 \begin{center}
  Instructor: Frank Wood (10:35-11:50)
 \end{center}
%%\studentinfo
\instructions{
  %\textbf{Circle your Instructor's Name along with the Lecture Time:}



  \begin{itemize}
  \item
    \textbf{Please show all your work.
            You may use back pages if necessary.}
  %\item
   % \textbf{Please put your \underline{simplified}
   %         final answers in the spaces provided.}
  \end{itemize}
}
\finishfirstpage

% Problems Start Here % ----------------------------------------------------- %


\problem{25}
{
Let $Y_i = \beta_0 + \beta_1 X_i + \epsilon_i$ be a linear regression model with distribution of error terms unspecified (but with mean $E(\epsilon) = 0$ and variance $V(\epsilon_i) = \sigma^2$ ($\sigma^2$ finite) known).  Show that $s^2 = MSE = \frac{\sum(Y_i-\hat Y_i)^2}{n-2}$ is an unbiased estimator for $\sigma^2$.  $\hat Y_i = b_0 + b_1 X_i$ where $b_0 = \bar Y - b_1 \bar X$ and $b_1 = \frac{\sum_i((X_i-\bar X)(Y_i - \bar Y))}{\sum_i(X_i-\bar X)}$
}
{
\vfill
  \answer
}
{
First, let's denote the followings:\\
$\hat{e_i}=y_i-\hat{y_i}$\\
$SXX=\displaystyle\sum_{i=1}^n (x_i-\bar{x})^2$\\
$SYY=\displaystyle\sum_{i=1}^n (y_i-\bar{y})^2$\\
$SXY=\displaystyle\sum_{i=1}^n (x_i-\bar{x})(y_i-\bar{y})$\\
\\
Now we set out to prove the following equation which accomplishes
essentially the final result:\\
$Var\hat{e_i}=E\hat{e_i}^2=(\frac{n-2}{n}+\frac{1}{SXX}(\frac{1}{n}\displaystyle\sum_{j=1}^nx_j^2+x_i^2-2(x_i-\bar{x})^2-2x_i\bar{x}))\sigma^2$\\
To prove the above, realize that:\\
\begin{align*}
Var(\hat{e_i})&=Var(y_i-\hat{\beta_0}-\hat{\beta_1}x_i)\\
              &=Var((y_i-\beta_0-\beta_1x_i)-(\hat{\beta_0}-\beta_0)-x_i(\hat{\beta_1}-\beta_1))\\
              &=Var(y_i)+Var(\hat{\beta_0})+x_i^2Var(\hat{\beta_1})-2Cov(y_i,\hat{\beta_0})-2x_iCov(y_i,\hat{\beta_1})+2x_iCov(\hat{\beta_0},\hat{\beta_1})\\
\end{align*}
The last equation holds because the covariance between any random
variable and a constant is zero, and all the $y_i$'s are independent
entailing that the $Cov(y_i,y_j)=0,i\not=j$\\
$Var(y_i)=\sigma^2$\\
Notice that(some algebras needed here, and the following tricks are
crucial in reducing the amount of calculation):\\
$\sum{(x_i-\bar{x})}=0$\\
$\beta_1=\frac{\sum{x_i-\bar{x}}y_i}{SXX}$\\
So now we have:\\
\begin{align*}
 Var(\beta_1)&=Var(\frac{SXY}{SXX})\\
             &=Var(\frac{\sum{(x_i-\bar{x})}y_i}{SXX})\\
             &=\frac{1}{SXX^2}\sum{x_i-\bar{x}}^2Var(y_i)\\
             &=\frac{\sigma^2}{SXX}\\
\end{align*}
And:\\
\begin{align*}
 Var(\beta_0)&=Var(\bar{y}-\hat{\beta_1}\bar{x})\\
             &=Var(\sum{(\frac{1}{n}-\frac{(x_i-\bar{x})\bar{x}}{SXX})y_i})\\
             &=\sum{(\frac{1}{n}-\frac{x_i-\bar{x}}{SXX}\bar{x})^2}\sigma^2\\
             &=\sum{[\frac{1}{n^2}+\frac{SXX*\bar{x}^2}{SXX^2}-\frac{2}{n}\frac{\bar{x}(x_i-\bar{x})}{XSS}]}\sigma^2\\
             &=[\frac{1}{n}+\frac{n\bar{x}^2}{SXX}]\sigma^2\\
             &=\frac{\sum{x_i}^2}{n*SXX}\sigma^2
\end{align*}
For the other terms in the decomposition of $Var(\hat{e_i})$, we
have:\\
\begin{align*}
 Cov(y_i,\hat{\beta_1})&=Cov(y_i,\frac{sum{x_i-\bar{x}}y_i}{SXX})\\
                       &=\frac{x_i-\bar{x}}{SXX}Var(y_i)\\
                       &=\frac{x_i-\bar{x}}{SXX}\sigma^2
\end{align*}
and:\\
\begin{align*}
 Cov(y_i,\hat{\beta_0})&=Cov(y_i,\bar{y}-\hat{\beta_1}\bar{x})\\
                       &=Cov(y_i,\frac{\sum{y_i}}{n}-\sum{(x_i-\bar{x})y_i}{SXX}\bar{x})\\
                       &=\frac{\sigma^2}{n}+\bar{x}\frac{x_i-\bar{x}}{SXX}\sigma^2
\end{align*}
At last, we have:\\
\begin{align*}
 Cov(\hat{\beta_0},\hat{\beta_1})&=Cov(\bar{y}-\hat{\beta_1}\bar{x},\hat{\beta_1})\\
                                 &=Cov(\frac{\sum{y_i}}{n}-\sum{\frac{(x_i-\bar{x})\bar{x}}{SXX}}y_i,\sum{\frac{(x_i-\bar{x})y_i}{SXX}})\\
                                 &=\displaystyle\sum_{i=i}^{n}(\frac{1}{n}-\frac{x_i-\bar{x}}{SXX}\bar{x})\frac{x_i-\bar{x}}{SXX}\sigma^2\\
                                 &=-\frac{\bar{x}}{SXX}\sigma^2
\end{align*}
Then plug in all the parts back to the decomposition of
$Var(\hat{e_i})$, we have:\\
$Var(\hat{e_i})=(\frac{n-1}{n}+\frac{1}{SXX}(\frac{1}{n}\displaystyle\sum_{j=1}^{n}{x_j}^2+x_i^2-2(x_i-\bar{x})^2-2x_i\bar{x}))\sigma^2$\\
\noindent Thus,\\
\begin{align*}
 E\hat{\sigma}^2&=\frac{1}{n}\displaystyle\sum_{i=1}^{n}E\hat{e_i}^2\\
                &=\frac{1}{n}\displaystyle\sum_{i=1}^{n}[\frac{n-2}{n}+\frac{1}{SXX}(\frac{1}{n}\displaystyle\sum_{j=1}^{n}x_j^2+x_i^2-2(x_i-\bar{x})^2-2x_i\bar{x})]\sigma^2\\
                &=[\frac{n-2}{n}+\frac{1}{nS_{xx}}\{\displaystyle\sum_{j=1}^{n}x_j^2+\displaystyle\sum_{i=1}^{n}x_i^2-2SXX-2\frac{1}{n}(\displaystyle\sum_{i=1}^{n}x_i)^2\}]\sigma^2\\
                &=(\frac{n-2}{n}+0)\sigma^2\\
                &=\frac{n-2}{n}\sigma^2\\
\end{align*}
where the third equation holds because:
$\sum{}{}x_i\bar{x}=\frac{1}{n}(\sum{}{} x_i)^2$\\
and the second to last equation holds since
$\sum{}{} x_i^2-\frac{1}{n}(\sum{}{} x_i)^2=SXX$\\
From the above equation, the result flows. \\
}

\problem{25}
{
Derive the maximimum likelihood estimators $\hat \beta_0, \hat \beta_1,$ and $\hat \sigma^2$ for parameters $\beta_0, \beta_1,$ and $\sigma^2$ for the normal linear regression model (i.e.~$\epsilon_i \sim_{iid} N(0,\sigma^2)$).
}
{
\vfill
  \answer
}
{
To figure the MLE of the parameters, we need to first write down
the likelihood function of the data, so under normal assumption, we
have the log-likelihood function as follows:\\
$logL(\beta_0,\beta_1,\sigma^2|x,y)=-\frac{n}{2}log(2\pi)-\frac{n}{2}log\sigma^2-\frac{\displaystyle\sum_{i=1}^{n}(y_i-\beta_0-\beta_1x_i)^2}{2\sigma^2}.$\\
For any fixed value of $\sigma^2$, $logL$ is maximized as a function
of $\beta_0$ and $\beta_1$, that minimize
\begin{eqnarray}
\displaystyle\sum_{i=1}^{n}(y_i-\beta_0-\beta_1x_i)^2
\end{eqnarray}
But to minimize this function is just to principle behind LSE, so
it's apparent that the MLE of $\beta_0$ and $\beta_1$ are the same
as their LSE's. Now, substituting in the log-likelihood, to find the
MLE of $\sigma^2$ we need to maximize\\
\[-\frac{n}{2}log(2\pi)-\frac{n}{2}log\sigma^2-\frac{\displaystyle\sum_{i=1}^{n}(y_i-\hat{\beta_0}-\hat{\beta_1}x_i)^2}{2\sigma^2}\]\\
This maximization problem is nothing but MLE of $\sigma^2$ in
ordinary normal sampling problems, which is easily given as\\
\[\hat{\sigma}^2=\frac{1}{n}\displaystyle\sum_{i=1}^{n}(y_i-\hat{\beta_0}-\hat{\beta_1}x_i)^2\]\\
If you are not familiar with the MLE in normal sampling setting, you
can take derivative with respect to $\sigma^2$ (N.B. not $\sigma$),
and then set the derivative to be zero. The solution of the equation
is just the MLE of $\sigma^2$.\\
}

\problem{50}
{
Do problem 1.19 in the book.
}
{
\vfill
  \answer
}
{
(a)\\
$data <- read.table("f:/TA1.txt")$\\
$attach(data)$\\
$x <- data[,2]$\\
$y <- data[,1]$\\
$SXX <- sum((x-mean(x))^2)$\\
$SYY <- sum((y-mean(y))^2)$\\
$SXY <- sum((x-mean(x))*(y-mean(y)))$\\
$beta.hat <- SXY/SXX$\\
$alpha.hat <- mean(y)-beta.hat*mean(x)$\\
We get the result the the LSE of the intercept and the slope are $2.11$ and $.038$.\\
The corresponding regression line is thus\\
\begin{eqnarray}
Y=.038+2.11X
\end{eqnarray}
(b)
\begin{figure}
\centering
\includegraphics[scale=.4]{TA1.png}
\end{figure}
From the graph, we can see that the regression passes through the center of the major part of the data, but does not capture all the features of the data.\\
(c)\\
Plug $x=30$ into (2), and the result is the point estimation of mean GPA with ACT score being 30.\\
\begin{center}
$\hat{y}=\hat{\alpha}+\hat{\beta}*x=2.11+.038*30=3.28$
\end{center}
(d)\\
It is nothing but the slope of the estimated regression line, since
the slope can be interpreted as the average GPA will increase by
.038 when the ACT score is enhanced by one point.\\
}


% Problems End Here % ------------------------------------------------------- %

\problemsdone
\end{document}

% End of the Document
