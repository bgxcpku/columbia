% Essential Formatting

\documentclass[12pt]{article}
\usepackage{verbatim}
\usepackage{epsfig,amsmath,amsthm,amssymb}
%\usepackage[questions, answersheet]{urmathtest}[2001/05/12]
%\usepackage[answersheet]{urmathtest}[2001/05/12]
\usepackage[answers]{urmathtest}[2001/05/12]


% For use with pdflatex
% \pdfpagewidth\paperwidth
% \pdfpageheight\paperheight

% Basic User Defs

\def\ds{\displaystyle}

\newcommand{\ansbox}[1]
{\work{
  \pos\hfill \framebox[#1][l]{ANSWER:\rule[-.3in]{0in}{.7in}}
}{}}

\newcommand{\ansrectangle}
{\work{
  \pos\hfill \framebox[6in][l]{ANSWER:\rule[-.3in]{0in}{.7in}}
}{}}

% Beginning of the Document

\begin{document}
\examtitle{LINEAR REGRESSION MODELS W4315}{HOMEWORK 2}{09/16/2009}
 \begin{center}
  Instructor: Frank Wood (10:35-11:50)
 \end{center}
%%\studentinfo
\instructions{
  %\textbf{Circle your Instructor's Name along with the Lecture Time:}



  \begin{itemize}
  \item
    \textbf{Please show all your work.
            You may use back pages if necessary.}
  %\item
   % \textbf{Please put your \underline{simplified}
   %         final answers in the spaces provided.}
  \end{itemize}
}
\finishfirstpage

% Problems Start Here % ----------------------------------------------------- %


\problem{25} { Problem 2.4 in the textbook on page 90}
 { \vfill
  \answer
} {
(a)\\
Following the code from problem 3 in homework 1, below is the R code for this problem:\\
%\begin{verbatim}
$y.hat <- alpha.hat+beta.hat*x$\\
$RSS <- sum((y-y.hat)^2)$\\
$MSE <- RSS/(120-2)$\\
$var.b1 <- MSE/SXX$\\
$sd.b1 <- sqrt(var.b1)$\\
%\end{verbatim}
From the code, we have standard error of $\beta_1$ is 0.0128.\\
Since t(.995,118)=2.618137, we have then the $\beta_1$'s .99 confidence interval is $.0.0388+/-2.618137*.0128$=(0.005287846,0.07231215).\\
It doesn't include zero. The reason that the director cares about the coverage of zero of CI is that he wants to be very much sure of if there is a positive relation between ACT score and GPA score.\\
(b)\\
Use the formula (2.20) to calculate the corresponding t-value. All the components are already known from the code above, so after plugging in the values, we have
$t^\ast=\frac{b_1-0}{sd(b_1)}=3.04$, and this value is greater than 2.618, so we reject the null hypothesis.\\
(c)\\
The P-value is $2P(t(118)>3.04)=.003$, which is smaller than .01, so we reject the null hypothesis. It's in sync with the result concluded from (b).
}

\problem{25} { Do problem 2.51 in the book.}
 { \vfill
  \answer
}
{
From (2.21), we have the explicit formula of $b_0$, so plugging in every term's formula we have the followings:\\
\begin{align*}
Eb_0&=E(\bar{Y})-Eb_1*\bar{X}\\
    &=\displaystyle\frac{1}{n}\sum_{i=1}^{n}EY_i-\frac{\bar{X}}{SXX}*E(SXY)\\
\end{align*}
where $EY_i=\beta_0+\beta_1*X_i$, and\\
\begin{align*}
E(SXY)&=E(\displaystyle\sum_{i=1}^{n}(X_i-\bar{X})Y_i)\\
      &=\displaystyle\sum_{i=1}^{n}(X_i-\bar{X})(\beta_0+\beta_1*X_i)
\end{align*}
Then we have:\\
\begin{align*}
Eb_0&=\displaystyle\frac{1}{n}\sum_{i=1}^{n}EY_i-\frac{\bar{X}}{SXX}*E(SXY)\\
    &=\displaystyle\frac{1}{n}\sum_{i=1}^{n}\beta_0+\beta_1*X_i-\displaystyle\sum_{i=1}^{n}(X_i-\bar{X})(\beta_0+\beta_1*X_i)\\
    &=\beta_0+\sum\frac{1}{n}-\frac{\bar{X}(X_i-\bar{X})}{SXX}(\beta_0-\beta_1X_i)\\
    &=\beta_0+\frac{nSXX-n\sum(X_i-\bar{X})X_i}{nSXX}\\
    &=\beta_0
\end{align*}
Thus, we proved that $b_0$ is an unbiased estimator of $\beta_0$.
}

\problem{50} { Problem 2.52 in the textbook on page 97}
 { \vfill
  \answer
}
{
(2.31) tells us that $\bar{Y}$ is independent of $b_1$. N.B. if 2 random variables X and Y and independent, then $Var(X+Y)=Var(X)+Var(Y)$.\\
So given the above result, we have the followings:\\
\begin{align*}
Var(b_0)&=Var(\bar{Y}-b_1\bar{X})\\
        &=Var(\bar{Y})+Var(b_1)*\bar{X}^2
\end{align*}
Since:\\
\begin{align*}
Var(\bar{Y})&=Var(\frac{1}{n}\sum Y_i)\\
            &=\frac{1}{n^2}\sum Var(Y_i)\\
            &=\frac{\sigma^2}{n}
\end{align*}
and $Var(b_1)=\frac{\sigma^2}{\sum(X_i-\bar{X})^2}$ by "Variance" on Page 43 of the textbook, so we have:\\
\begin{center}
$Var(b_0)=\sigma^2(\frac{1}{n}+\frac{\bar{X}^2}{\sum(X_i-\bar{X})^2})$
\end{center}
The above equation is a special case of (2.29b) in the sense that in (2.29b), if $X_h$ equals 0 or $2\bar{X}$ then it becomes (2.22b).
}


% Problems End Here % ------------------------------------------------------- %

\problemsdone
\end{document}

% End of the Document
