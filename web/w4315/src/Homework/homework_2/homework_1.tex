% Essential Formatting
   
\documentclass[12pt]{article}
\usepackage{epsfig,amsmath,amsthm,amssymb}
\usepackage[questions, answersheet]{urmathtest}[2001/05/12]
%\usepackage[answersheet]{urmathtest}[2001/05/12]
%\usepackage[answers]{urmathtest}[2001/05/12]

% For use with pdflatex
% \pdfpagewidth\paperwidth
% \pdfpageheight\paperheight

% Basic User Defs

\def\ds{\displaystyle}

\newcommand{\ansbox}[1]
{\work{
  \pos\hfill \framebox[#1][l]{ANSWER:\rule[-.3in]{0in}{.7in}}
}{}}

\newcommand{\ansrectangle}
{\work{
  \pos\hfill \framebox[6in][l]{ANSWER:\rule[-.3in]{0in}{.7in}}
}{}}

% Beginning of the Document

\begin{document}
\examtitle{LINEAR REGRESSION MODELS W4315}{HOMEWORK 1}{09/16/2009}
 \begin{center}
  Instructor: Frank Wood (10:35-11:50) 
 \end{center}
%%\studentinfo
\instructions{
  %\textbf{Circle your Instructor's Name along with the Lecture Time:}

 

  \begin{itemize}
  \item
    \textbf{Please show all your work.
            You may use back pages if necessary.}
  %\item
   % \textbf{Please put your \underline{simplified}
   %         final answers in the spaces provided.}
  \end{itemize}
}
\finishfirstpage

% Problems Start Here % ----------------------------------------------------- %


\problem{25}
{
Let $Y_i = \beta_0 + \beta_1 X_i + \epsilon_i$ be a linear regression model with distribution of error terms unspecified (but with mean $E(\epsilon) = 0$ and variance $V(\epsilon_i) = \sigma^2$ ($\sigma^2$ finite) known).  Show that $s^2 = MSE = \frac{\sum(Y_i-\hat Y_i)^2}{n-2}$ is an unbiased estimator for $\sigma^2$.  $\hat Y_i = b_0 + b_1 X_i$ where $b_0 = \bar Y - b_1 \bar X$ and $b_1 = \frac{\sum_i((X_i-\bar X)(Y_i - \bar Y))}{\sum_i(X_i-\bar X)^2}$
}
{
\vfill
  \answer
}
{
\begin{eqnarray}
 \frac{d}{dx})-x^2 + ln(x) -a) &=& 0 \\
 -2x+\frac{1}{x} &=& 0 \\
  2x^2 &=& 1 \\
  x &=& \frac{1}{2} \\
\end{eqnarray}
}

\problem{25}
{
Derive the maximimum likelihood estimators $\hat \beta_0, \hat \beta_1,$ and $\hat \sigma^2$ for parameters $\beta_0, \beta_1,$ and $\sigma^2$ for the normal linear regression model (i.e.~$\epsilon_i \sim_{iid} N(0,\sigma^2)$).
}
{
\vfill
  \answer
}
{
}

\problem{50}
{
Do problem 1.19 in the book.
}
{
\vfill
  \answer
}
{
}


% Problems End Here % ------------------------------------------------------- %

\problemsdone
\end{document}

% End of the Document
