\documentclass{beamer}
\usepackage{graphicx}
\usepackage{verbatim}
\usepackage{amsmath}
\usepackage{amsfonts}
\usepackage{setspace}
% \usepackage{beamerthemesplit} // Activate for custom appearance


\newcommand{\ubf}{\mathbf{u}}
\newcommand{\xbf}{\mathbf{x}}
\newcommand{\sbf}{\mathbf{s}}
\newcommand{\py}{\mathcal{PY}}
\newcommand{\vbf}{\mathbf{v}}
\newcommand{\Prob}{\mathrm{P}}
\newcommand{\Psmooth}{\Prob_\text{smooth}}
\newcommand{\parent}{\pi}
\newcommand{\suffix}{\sigma}
\newcommand{\UHPYP}{SM}
\newcommand{\PLUMP}{PLUMP}
\newcommand{\Oh}{\mathcal{O}}
\newcommand{\tree}{\mathcal{T}}

% \newcommand{\cusk}{c_{\ubf s k}}
% \newcommand{\cus}{c_{\ubf s \cdot}}
% \newcommand{\cu}{c_{\ubf \cdot \cdot}}
% \newcommand{\tus}{t_{\ubf s}}
% \newcommand{\tu}{t_{\ubf \cdot}}
\newcommand{\cusk}{c_{\ubf s k}}
\newcommand{\cus}{c_{\ubf s}}
\newcommand{\cu}{c_{\ubf \cdot}}
\newcommand{\tus}{t_{\ubf s}}
\newcommand{\tu}{t_{\ubf \cdot}}
\newcommand{\cset}{\{\cusk\}_{s\in \Sigma,k \in \{1,\ldots,t_{\ubf s}\}}}
\newcommand{\tset}{\{\tus\}_{s\in \Sigma}}
\newcommand{\bydef}{\equiv}
\newcommand{\state}{\mathcal{S}_{\xbf}}
\newcommand{\statei}{\mathcal{S}_{\xbf_{1:i}}}
\newcommand{\emptystring}{\varepsilon}
\newcommand{\gcount}{\hat{c}}
\newcommand{\escape}{\mathtt{esc}}

\newcommand{\todo}[1]{\begin{center}\textbf{TODO: } #1 \end{center}}
\newcommand{\figref}[1]{\figurename~\ref{#1}}
\newcommand{\predictive}{\Prob(x_i|\xbf_{1:i-1})}
\newcommand{\ywcomment}[1]{\textbf{#1}}
\newcommand{\jgcomment}[1]{ { \textcolor{red}{#1} } }




\title{\BNPfull}
\author{Presenter: Dr. Frank Wood}
\date{\today}


\begin{document}

\frame{\titlepage}


\frame[t] {
 \frametitle{\BNPfull ( \BNP)}
\begin{block}{Motivation}
\begin{itemize}
\item Never know the full data generating mechanism
\begin{itemize}
\item Want to make the most general assumptions.
\item Guard against possible gross model misspecification.
\end{itemize}
\end{itemize}
\end{block}
\begin{block}{\BNPfull}
\begin{itemize}
\item Parameters can be described by functions or other infinite dimensional objects
\begin{itemize}
\item Cumulative distribution function (CDF)
\item Density function.
\item Nonparametric regression function
\item Unknown link function in generalized linear model (GLM)
\end{itemize}
\end{itemize}
\end{block}
}

\frame[t] {
 \frametitle{\BNP}
\begin{block}{Challenges}
\begin{itemize}
\item Construction of prior distribution involves specifying appropriate probability measure on function spaces.
\item Knowledge of minute details of such distributions typically not available.
\end{itemize}
\end{block}

\begin{block}<2>{Standard Approach}
\begin{itemize}
\item Prior typically chosen for computational practicality
\item Key prior parameters might be chosen subjectively
\item Important: prior should have large support 
\item Large support of the prior helps the posterior distribution to have good frequentist properties in large samples.
\end{itemize}
\end{block}
}

\frame[t] {
 \frametitle{\BNP Posterior consistency}
\begin{block}{{\em Posterior consistency}}
\begin{itemize}
\item Basic frequentist validation of a Bayesian estimation procedure
\item In limit of infinite data, does the posterior distribution converge to the true underlying parameter?
\item Lack of consistency is {\em undesirable}.
\item Rate of convergence can be used to distinguish different estimation procedures 
\begin{itemize}
\item How quickly can a ball around the true value shrink while retaining almost all of the posterior probability?
\end{itemize}

\end{itemize}
\end{block}

\only<2>{\em Posterior consistency of \BNP models is an area of active research}

}

\frame[t] {
 \frametitle{\BNP models}
\begin{block}{Examples}
\begin{itemize}
\item Dirichlet process 
\begin{itemize}
\item Mixtures of Dirichlet processes \cite{Antoniak1974, MacEachern1998}
\item Dirichlet process mixture \cite{Escobar1995a}
\end{itemize}
\item Gaussian process
\begin{itemize}
\item Gaussian processes for machine learning (book) \cite{Rasmussen2006}
\end{itemize}
\item Indian Buffet process, Chinese restaurant process, Beta process, Dependent Dirichlet process, Hierarchical Dirichlet Process (HDP), HDP-HMM, HDP-LDA, Sequence Memoizer, etc.
\end{itemize}
\end{block}
}

\frame[t] {
 \frametitle{\BNP models}
 \begin{block}{To Start}

\begin{itemize}
\item Look at role of Dirichlet process
\item Discuss most important properties
\item Informally talk about posterior convergence in such models
\end{itemize}
\end{block}

}

\frame[t] {
 \frametitle{The Dirichlet Process}
 \begin{block}{Motivation}
Say we'd like to estimate a probability measure (or CDF) on the real line, with i.i.d.~ observations from it where the CDF is completely arbitrary.
\end{block}
 \begin{block}<2-3>{Classical approach}
Build a nonparametric estimate of the CDF directly from the observations.
\end{block}
 \begin{block}<3-3>{Bayesian approach}
Need a prior for the CDF (or for a random probability measure) and methods (algorthims, etc.) to estimate the posterior distribution.
\end{block}

}

\frame[t] {
 \frametitle{Closest Parametric Analog}
 \begin{block}{Multinomial / Dirichlet}
\begin{itemize}
\item  Multinomial distribution specifies an arbitrary probability distribution on the sample space of finitely many integers.
\item Multinomial model can be derived from an arbitrary distribution by grouping the data in finitely man categories.
\item Formalism
\begin{itemize}
\item Let $(\pi_1, \ldots, \pi_k)$ be the probabilities of the categories with frequencies $n_1, \ldots, n_k$.  The multinomial likelihood is proportional to $\pi_1^{n_1}, \ldots, \pi_k^{n_k}$.
\item The finite-dimensional Dirichlet prior has density proportional to $\pi_1^{c_1-1}, \ldots, \pi_k^{c_k-1}$
\item The posterior has density proportional to $\pi_1^{n_1+c_1-1}, \ldots, \pi_k^{n_k+c_k-1}$ which is again Dirichlet.
\end{itemize}

\end{itemize}

 
\end{block}
}

\frame[t] {
 \frametitle{Definition }
 The Dirichlet {\em process} is a probability distribution on the space of probability measures which induces finite-dimensional Dirichlet distributions when the data are grouped.

\begin{itemize}
\item For any measureable partition $\{B_1, \ldots B_k\}$ of $\Real$ the probability vector $(P(B_1),\ldots,P(B_k))$ is a finite-dimensional Dirichlet distribution.
\item This means that the parameters of the finite-dimensional Dirichlet dist.~must be special.
\item For instance, the joint distribution of r $(P(B_1),\ldots,P(B_k))$ must agree with the joint distribution $(P(A_1),\ldots,P(A_k))$ when $\{A_1, \ldots A_k\}$ is finer than $\{B_1, \ldots B_k\}$  since for any $i$, $P(B_i)$ would be the sum of some $P(A_j)$
\end{itemize}
}

\frame[t] {
 \frametitle{Definition }

A finite-dimensional Dirichlet distribution property is that summing the probabilities of different partitions gives rise to a new Dirichlet distribution whose parameters corresponding to the summed partitions are added.  Let $\alpha(B)$ be the parameter corresponding to $P(B)$ in the specified Dirichlet joint distribution, it follows that $\alpha(\cdot)$ must be an additive set function.
\bigskip

Let $\alpha$ be a finite measure on a given Polish space $\mathfrak{X}$.  A random measure $P$ on $\mathfrak{X}$ is called a Dirichlet process if for every finite measureable partition $\{B_1, \ldots, B_k\}$ of $\mathfrak{X}$, the joint distribution of $(P(B_1),\ldots,P(B_k))$ is a $k$-dimensional Dirichlet distribution with parameters $\alpha(B_1),\ldots,\alpha(B_k)$
\bigskip

We call $\alpha$ the base measure of the Dirichlet process and denote the corresponding Dirichlet process $\mathcal{D}_\alpha$.
}

\frame[t] {
 \frametitle{A Little Problem }

Even when $\alpha$ is a measure, it still isn't clear that $P$ is a probability measure, i.e.~that it sums to one.
\bigskip

Several strategies could be taken towards demonstrating this, we will call them various constructions of the DP. 
\begin{itemize}
\item Naive
\item Countable generator
\item Normalization
\end{itemize}

}

 Can we use Kolmogorov's Existence Theorem ? (in short, no)


\bibliographystyle{apalike}
\frame[t] {
 \frametitle{Bibliography}
\bibliography{../../../../papers/uber.bib}
}

\end{document}
