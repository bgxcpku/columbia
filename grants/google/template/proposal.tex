% proposal.tex
\documentclass[10pt]{article}
% preamble.tex

% enclose any latex in \comment{} to suppress it
\newcommand{\comment}[1]{}
\newcommand{\citeNP}[1]{\cite{#1}}

\usepackage{graphicx}
\usepackage{multicol}
\usepackage{times}
%\usepackage{floatflt}
\graphicspath{{./}{figures/}}

% NO psfig -- pdflatex does not support it ...
% \usepackage{psfig}
% \psfigurepath{./:figures/}
% \DeclareGraphicsRule{.eps.gz}{eps}{.eps.bb}{'gunzip -c #1}

\newcommand{\upline}{\vspace*{-\baselineskip}}
\newcommand{\up}{\vspace*{-6pt}}
\newcommand{\downline}{\vspace*{\baselineskip}}
\newcommand{\sep}{~~~~~~~~~~}

% this hardcodes the bib name, which we don't want to do.
% \renewcommand{\thebibliography}[1]{\section*{References Cited}
% \addcontentsline{toc}{section}{References Cited}\list
%  {[\arabic{enumi}]}{\settowidth\labelwidth{[#1]}\leftmargin\labelwidth
%  \advance\leftmargin\labelsep
%  \usecounter{enumi}}
%  \def\newblock{\hskip .11em plus .33em minus -.07em}
%  \sloppy\clubpenalty4000\widowpenalty4000
%  \sfcode`\.=1000\relax}

% fix this to specify width and height, and solve for the margins
% margins.tex

% import calc
\usepackage{calc}

%
\newlength{\myrightmargin}
\newlength{\myleftmargin}
\newlength{\mytopmargin}
\newlength{\mybottommargin}

% Change these settings to change the margins
\setlength{\myrightmargin}{1.0in}
\setlength{\myleftmargin}{1.0in}
\setlength{\mytopmargin}{0.75in}     
\setlength{\mybottommargin}{0.75in} 
\setlength{\oddsidemargin}{0.0in}   % extra room on inside side

%%% use margin settings to set width variables
\setlength{\evensidemargin}{0 in}
\setlength{\marginparsep}{0 in}
\setlength{\marginparwidth}{0 in}
\setlength{\hoffset}{\myleftmargin - 1.0in}
\setlength{\textwidth}
  {8.5in -\myleftmargin -\myrightmargin -\oddsidemargin}

%%% use margin settings to set height variables
\setlength{\voffset}{\mytopmargin -1.0in}
\setlength{\topmargin}{0 in}
\setlength{\headheight}{12 pt}
\setlength{\headsep}{20 pt}
\setlength{\footskip}{36 pt}
\setlength{\textheight}
  {11.0in-\mytopmargin-\mybottommargin-\headheight-\headsep-\footskip}

% \oddsidemargin 0.2cm
% \evensidemargin 0cm
% \textwidth 16.0cm
% \topmargin -1.25cm
% \textheight 22.94cm

% remove parindent, squeeze grafs
\setlength{\parindent}{0in}
\setlength{\parskip}{1ex}

\def\nibf#1{\noindent\textbf{#1}}


\date{\today }

% author.tex
\author{
PI(s)/Author(s)\\
Affiliation(s)\\
Address(es)}


% title.tex
\title{NSF Proposal Template\\
(\LaTeX\ format)}


\begin{document}

\bibliographystyle{plain}
\setcounter{page}{1}

\maketitle

% checklist.tex
This latex template includes all major pieces (in bold below) typically
required for an NSF proposal.  The idea is that the PIs create one PDF
document, from which each piece can be extracted and uploaded to FastLane.
I have structured it as a two-PI proposal, to make things easy in case
(for example) each PI wishes to write a separate technical section or 
bring a separate \verb+.bib+ file to the bibliography.

You can retrieve this template from any CSAIL machine with the following
command (all on one line):
\begin{verbatim}
svn checkout
svn+ssh://svn.csail.mit.edu/afs/csail.mit.edu/group/rvsn/funding/NSF/template
\end{verbatim}

Using this template might save both you and CSAIL HQ staff some work
next time you put in an NSF proposal.

{\hfill {\small -- Seth Teller (\verb+teller@csail.mit.edu+), Nov. 2006}}
%
\begin{itemize}

\item {\bf PI/Co-PI Information, Proposal Title, etc.} must be keyed in
directly to FastLane.

\item \verb+proposal.tex+ -- this document.

\item \verb+author.tex, title.tex+ -- {\bf Proposal Cover Page}\\
      (not uploaded, but admin/HQ folks can refer to this 
       when keying in FastLane fields for proposal creation).

\item \verb+summary.tex+ -- {\bf Proposal Summary} (1 page).

\item \verb+objectives.tex, technical.tex, sow.tex, related.tex+ -- \\
      {\bf Research Objectives, Statement of Work, Related Work} (15 pages).

\item \verb+refsA.bib, refsB.bib+ -- {\bf References Cited} (no page limit).

\item \verb+budget.tex+ -- {\bf Budget Justification}, written by PIs,
      massaged by HQ staff and uploaded to Fastlane.

\item \verb+facilities.tex+ -- {\bf Facilities Description}, written by PIs,
      massaged by HQ staff, uploaded to Fastlane.

\item \verb+biosketches.tex+ -- {\bf Biographical sketches} 
      (2 pages per PI; no bios needed for co-I's).

\item {\bf Budget spreadsheet} -- prepared by CSAIL HQ staff.

\item {\bf Current and Pending Support page} -- prepared by CSAIL HQ staff.

\item {\bf Supplementary documents} -- things such as quotes for
  large equipment items.\\ PIs/Admins or CSAIL HQ staff should upload these separately 
  to FastLane as needed.

\item {\bf Suggested Reviewers / Non-Reviewers} -- optional; key in directly to FastLane.

\item {\bf Deviation authorizations} and {\bf Additional
  single-copy documents} -- usually N/A so are omitted here.

\end{itemize}

\newpage
\pagestyle{empty}

% 1 page summary, required by NSF
% squinch space a bit if needed
% \setlength{\parindent}{0in}
% \setlength{\parskip}{0.5ex}
\section*{Project Summary {\small (1 page)}}
\addcontentsline{toc}{section}{A. Project Summary}
%
This is a summary of the project, with a strict 1-page limit.

After generating PDF, strip out this page and upload it
separately to FastLane.

{\bf Intellectual Merits of the Proposal Activity} 

Address these here.

{\bf Broader Impacts of the Proposal Activity} 

Address these here.




\newpage
\pagestyle{plain}
\setcounter{page}{1}

% GPG dictates section, so must use \subsection inside
\section*{Project Description}  %% limited to 15 pages 
\addcontentsline{toc}{section}{C. Project Description}

%% {\em The main body of the proposal should be a clear statement of
%% the work to be undertaken and should include: objectives for the
%% period of the proposed work and expected significance, relation to
%% longer-term goals of the PI's project, and relation to the present
%% state of knowledge in the field, to work in progress by the PI
%% under other support and to work in progress elsewhere.  The
%% statement should outline the general plan of work, including 
%% the broad design of activities to be undertaken, an adequate
%% description of experimental methods and procedures, and if
%% appropriate, plans for preservation, documentation, and sharing of
%% data, samples, physical collections and other related research
%% products.  Any substantial collaboration with individuals not
%% included in the budget should be described with a letter from 
%% each \ldots.} 

% a top-level Section, with Subsections
% objectives.tex
\section{Research Objectives}
\label{objectives}
%
This is the first section of the 15-page technical portion of the
proposal.  Use citations liberally; the bibliography doesn't count
towards the page limit.

State research objectives here.

\subsection{Intellectual Merit of the Proposed Work}
%
Describe the intellectual merit of the proposed work here.

\begin{figure}[h]
\centerline{\mbox{\includegraphics[width=3in]{cycle}}}
\caption{Our proposed research plan involves an iterated 
         cycle of design, implementation and user testing.}
\label{fig:cycle}
\end{figure}

Describe the intellectual merit of the proposed work here.

\subsection{Broader Impact of Proposed Work}
%
Describe the broader impact of the proposed work here.

Describe the broader impact of the proposed work here.


% more technical content
% technical.tex
\section{Another Technical Section}
\label{sec:technical}
% 
This is another technical section.

\subsection{Technical Subsection}
%
This is a technical subsection.

\subsection{Technical Subsection}
%
This is a technical subsection.



% statement of work
% plan.tex
\section{Statement of Work}
\label{plan}
%
Give the statement of work here.  Give the statement of work here.  Give
the statement of work here.  Give the statement of work here.  Give the
statement of work here.  Give the statement of work here.  Give the
statement of work here.  Give the statement of work here.  Give the
statement of work here.

\subsection{Schedule}
%
\bigskip
\begin{tabular}{l|l}
\hspace{.25in} Year 1 \hspace{.25in} & 
\begin{minipage}{5in}
  Some pieces\\
  Some other pieces\\
  \\
  By the end of year 1, we expect to ...
\medskip
\end{minipage}\\
\hline
\hspace{.25in} Year 2 & 
\begin{minipage}{5in}
  \medskip
  Some pieces \\
  Some other pieces \\
  \\
  By the end of year 2, we expect to ...
\medskip
\end{minipage}\\
\hline
\hspace{.25in} Year 3 & 
\begin{minipage}{5in}
  \medskip
  Some pieces \\
  Some other pieces \\
  \\
  By the end of year 3, we expect to ...
\end{minipage}
\end{tabular}

\subsection{Team Composition and Expertise} 
\label{expertise} 
%
Our team comprises four investigators, A, B, C, and D, whose areas of
expertise span the critical areas of ...

A's areas of expertise include ...

B's areas of expertise include ...

\subsection{Connections to Education and Outreach}
\label{education}
%
Describe any connections to educational and outreach activities here.
Describe any connections to educational and outreach activities here.
Describe any connections to educational and outreach activities here.
Describe any connections to educational and outreach activities here.
Describe any connections to educational and outreach activities here.
Describe any connections to educational and outreach activities here.
Describe any connections to educational and outreach activities here.
Describe any connections to educational and outreach activities here.


% related work
% significance.tex
\section{Related Work}
%
Describe related work \cite{bar,foo} here.


% prior NSF work by PI's
% priornsf.tex
\section{Results from Prior NSF Support}
\label{priornsf}
%
Describes any prior NSF-funded work by the PI(s) here.


\noindent
Person A has served as a PI on the following NSF awards:

\noindent
{\bf NSF XXX-YYYYYYY}, TITLE \\
(1995--1998); PI: A

This project involved ...


\noindent
{\bf NSF XXX-YYYYYYY}, TITLE \\
(1996-2001); PI's: A, B, C

This project involved ...

\noindent
Person B has served as a PI or Co-I on the following NSF awards:

etc.


% turn off pagination; no page limit for references
\pagestyle{empty}
\renewcommand{\refname}{References Cited}
% generate References Cited section
\bibliography{refsA,refsB}

\newpage

% description of facilities
\section{Facilities}
\label{facilities}
%
All work will be performed within the MIT Computer Science and
Artificial Intelligence Laboratory (CSAIL).  CSAIL is equipped with a
large number of individual workstations, running a variety of operating
systems, all interconnected on wireless and wired networks.  Most
machines use AFS for file storage; the CSAIL administrators provide an
SVN server for source control of multiple source trees.

This project will have access to specific MIT CSAIL facilities,
including:
\begin{itemize}

\item Robotics laboratory

\item Hardware lab

\item System administrators

\item Machine shop

\item Gigabit Ethernet

\item Several Tb of AFS storage with daily backup

\end{itemize}

CSAIL also maintains extensive mechanical and electrical prototyping
facilities, allowing rapid system construction. A complete machine shop
and sheet metal prototype fabrication facility is available, along with
stocks of small electrical components, connectors, wire, etc. All of
this enables us to make quick mock-ups of proposed designs.  The
Laboratory provides an environment of excellent students interested 
in advanced systems, networking HCI and software technologies,
knowledge-representation, reasoning, planning, decision-theoretic
techniques, and inference.


\newpage

% budget justification
% budget.tex
% defs for below.  note " " after defs.
\def\piA{Person A }
\def\piB{Person B }
\def\numRAs{2 }
\def\pilist{PIs: \piA and \piB}
\def\thisyear{2006}
\def\startdate{06/01/2007 }
\def\enddate{05/31/2010 }
\def\phdstipend{\$2,125/mo }
\def\mscstipend{\$1,940/mo }
\def\studraise{4\% }
\def\salaryallocationrate{8.34\%} % no trailing space
\def\employeebenefitrate{27.0\% }
\def\vacationaccrualrate{9.5\% }
\def\networkservicescost{\$100/person/month }
\def\networkfacilitycharges{\$150/person/month} % no trailing space
\def\MITninemonthtuition{\$33,400} % no trailing space
\def\MITtuitioninflator{4\% }
\def\MITtuitionsubsidy{45\% }
\def\MITtuitioncharge{55\% }
\def\msAllocation{1.24\%}
\def\facostinception{07/01/2006} % no trailing space
\def\facostrate{65\% }

% make the outer enumerated list alpha A, B, C ...
\renewcommand{\theenumi}{\Alph{enumi}}
\renewcommand{\theenumii}{\arabic{enumii}}

\vspace*{1.0in}
\begin{center}
{\large
MIT / Computer Science and Artificial Intelligence Laboratory \\
\pilist \\
Proposed Budget Period: \startdate -- \enddate \\
Budget Justification for Cost Proposal
}
\end{center}

\begin{enumerate}

%A
\item \underline{Key Personnel:}

\begin{tabular}{|l|l|} \hline
\underline{Last Name} & \underline{Review/Raise} \\ \hline
\piA & June \\ \hline
\piB & June \\ \hline
\end{tabular}

MIT fully supports the academic year salaries of professors, associate
professors, and assistant professors, but makes no specific commitment
of time or salary to any individual research project.

%B
\item \underline{Other Personnel:}

\begin{enumerate}
  \item \underline{Research Assistants:} \\{~}\\
100\% of the stipend is charged to the research project. The RA stipend
is not subject to employee benefits. Stipend for the year beginning on
\startdate is \phdstipend for a PhD student and \mscstipend for a
Masters student. A \studraise raise is applied each year (in June).

%2
  \item \underline{Other (Technical \& Administrative Support):}\\{~}\\
The Computer Science Artificial Intelligence Laboratory (CSAIL) provides administrative
services for all principal investigators who submit proposals through CSAIL. These
administrative services are run by the Headquarters Staff and include Fiscal, Personnel,
Facilities and other CSAIL operations.\\{~}\\
These services are supported by an Allocated Project Level Cost, which is assessed against all
contracts and grants. The current rate for the Salary Allocation is \salaryallocationrate. The Allocation
Base is shown below:

\begin{tabular}{|c|c|c|c|} \hline
Allocation & Year 1 & Year 2 & Year 3 \\ \hline
Base & \$XXX,YYY & \$XXX,YYY & \$XXX,YYY \\ \hline
\end{tabular}

\end{enumerate}

\item \underline{Fringe Benefits}

\begin{enumerate}

%(a)
\item Employee benefits are calculated at the rate of \employeebenefitrate and 
are applied to total salary expenses, less Research Assistants.

\item Vacation accruals are calculated at the rate of \vacationaccrualrate and 
are applied to total salary expenses, less Faculty and Research
Assistants.

\end{enumerate}

%D
\item \underline{Travel:}

\begin{enumerate}

\item \underline{Domestic Travel:}

\item \underline{Foreign Travel:}

\end{enumerate}

\item \underline{Other Direct Costs:}

\begin{enumerate}

\item \underline{Material \& Supplies:}\\{~}\\
Estimated costs for software and supplies needed for the project.

\item \underline{Computer Services:}\\{~}\\
MIT/CSAIL has a centralized network services function. The costs consist of Network Services at
\networkservicescost and Network Facility Charges at \networkfacilitycharges. The base number of
people used for this calculation was \numRAs (the two full time RAs).

\item \underline{Other:}\\{~}\\
(a) RA tuition: For the academic year starting \thisyear, MIT 9-month
tuition is \MITninemonthtuition. A \MITtuitioninflator annual inflator
is applied each year. MIT will subsidize \MITtuitionsubsidy of tuition,
leaving \MITtuitioncharge to be charged to the project. During the
summer, MIT has waived tuition.\\{~}\\ 
(b) Allocated expenses are assessed against all contracts and
grants. The current rate for the Materials and Services Allocation is
\msAllocation. These funds help support the Headquarters staff mentioned
above in the section entitled ``Other (Technical \& Administrative
Support)''. Please see the table in that section for the allocation
base.

\item \underline{Equipment:}

The equipment line items will support purchases as follows:

\begin{itemize}

\item Year One (\$X,XXX): enter purpose, items and costs here

\item Year Two (\$X,XXX): enter purpose, items and costs here

\item Year Three (\$X,XXX): enter purpose, items and costs here

\end{itemize}

\item \underline{[Describe any other direct cost items here]:}

\end{enumerate}

\item \underline{Indirect Costs (Facilities \& Administrative Costs):}\\{~}\\
Effective \facostinception, F\&A Costs are calculated by applying the
negotiated rate of \facostrate to the Modified Total Direct Cost (MTDC)
base. The MTDC base includes all direct costs, except Graduate Student
Tuition, Network Facilities Charges, the Salary Allocation (and
associated benefits), and the Materials and Services Allocation.

\end{enumerate}


\newpage

% bios for all Co-PI's (not required for Co-I's)
\section*{PI Biography: Person A}
%
{\nibf{Professional Preparation:}}
%
\begin{verbatim}
School                              Degree                     Date
\end{verbatim}

\nibf{Appointments}

\begin{verbatim}
Organization                        Position                   Date
\end{verbatim}

\nibf{Five Publications Relevant to Proposal}

\begin{verbatim}
1. Authors,
   Title,
   Booktitle,
   Location, Month, Year, Pages.
2. Authors,
   Title,
   Venue,
   Location, Month, Year, Pages.
\end{verbatim}

\nibf{Five Other Selected Publications}

\begin{verbatim}
1. Authors,
   Title,
   Booktitle,
   Location, Month, Year, Pages.
2. Authors,
   Title,
   Booktitle,
   Location, Month, Year, Pages.
\end{verbatim}

\nibf{Synergistic Activities}

\begin{verbatim}
1. Activity 1.
2. Activity 2.
\end{verbatim}

\nibf{Awards and Honors}

\begin{verbatim}
Award                                       Date
\end{verbatim}

\nibf{Recent Collaborators}
\begin{verbatim}
Foo (MIT), Bar (Harvard)
\end{verbatim}

\nibf{Graduate Advisor}

\begin{verbatim}
Person C (School)
\end{verbatim}

\nibf{Graduate Students (Completed)}

\begin{verbatim}
Person E (Now at Nokia)
Person F (Now at CalTech)
\end{verbatim}

\newpage

\section*{PI Biography: Person B}

etc.



\end{document}
